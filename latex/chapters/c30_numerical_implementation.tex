% !TEX root = ../main.tex
\chapter{Numerical Implementation of Two-Dimensional Electronic Spectroscopy}
\label{chap:numerical_implementation}

\noindent
This chapter presents the computational framework developed for simulating two-dimensional electronic spectroscopy (2DES) spectra of molecular systems. The implementation focuses on the technical aspects of quantum system evolution, pulse sequence generation, data processing, and visualization, providing a bridge between the theoretical framework established in previous chapters and practical computational methods.

\section{Overview of Simulation Architecture}
\label{sec:simulation_architecture}

\noindent
The simulation framework follows a modular structure that separates quantum mechanical evolution from spectroscopic data processing. The general workflow consists of:

\begin{enumerate}
	\item Definition of quantum system parameters (energy levels, dipole moments, coupling strengths)
	\item Generation of pulse sequences with specific time delays and phases
	\item Evolution of the density matrix under pulse influences and environmental coupling
	\item Calculation of the system's polarization response
	\item Processing of time-domain data to obtain frequency-domain 2D spectra
	\item Averaging over inhomogeneous distributions and phase cycling
\end{enumerate}

\noindent
This modular approach enables systematic investigation of different physical parameters while maintaining computational efficiency through parallel processing capabilities.

\section{Computing Two-Dimensional Polarization Response}
\label{sec:computing_polarization}

\noindent
The core calculation computes the third-order polarization response by iterating through coherence times ($t_{\text{coh}}$) and detection times ($t_{\text{det}}$) for a given waiting time ($T_{\text{wait}}$):

\begin{equation}
	\label{eq:polarization_third_order}
	P^{(3)}(t_{\text{coh}}, T_{\text{wait}}, t_{\text{det}}) = \Tr[\hat{\mu} \cdot \hat{\rho}^{(3)}(t_{\text{coh}}, T_{\text{wait}}, t_{\text{det}})]
\end{equation}

\noindent
where $\hat{\mu}$ is the dipole operator and $\hat{\rho}^{(3)}$ is the third-order density matrix. This calculation implements the three-pulse sequence through sequential evolution steps:

\begin{enumerate}
	\item Apply the first pulse with phase $\phi_0$ and evolve for coherence time $t_{\text{coh}}$
	\item Apply the second pulse with phase $\phi_1$ and evolve for waiting time $T_{\text{wait}}$
	\item Apply the third pulse with phase $\phi_2$ and measure response during detection time $t_{\text{det}}$
\end{enumerate}

\noindent
The temporal evolution utilizes either custom equations derived from literature or standard quantum dynamics solvers from the QuTiP \cite{lambertetal2024qutip5quantum} library, ensuring both flexibility and computational accuracy.

\section{Inhomogeneous Broadening Implementation}
\label{sec:inhomogeneous_broadening}

\noindent
To accurately model molecular systems, the simulation accounts for inhomogeneous broadening by averaging over a distribution of transition frequencies. The distribution follows a Gaussian profile:

\begin{equation}
	\label{eq:gaussian_distribution}
	\sigma(E - E_0) = \frac{1}{\sigma_{\text{val}}\sqrt{2\pi}} \exp\left(-\frac{(E-E_0)^2}{2\sigma_{\text{val}}^2}\right)
\end{equation}

\noindent
where $\sigma_{\text{val}} = \Delta/(2\sqrt{2\ln{2}})$ relates the standard deviation to the full width at half maximum $\Delta$.

\noindent
The sampling from this distribution employs a rejection sampling algorithm that ensures accurate representation of the broadening profile while maintaining computational efficiency. The algorithm operates by:

\begin{enumerate}
	\item Defining the sampling range $[E_{\text{min}}, E_{\text{max}}] = [E_0 - E_{\text{range}} \cdot \Delta, E_0 + E_{\text{range}} \cdot \Delta]$
	\item Computing the maximum value $\sigma_{\text{max}}$ of $\sigma(E)$ in the range
	\item Generating candidate values uniformly and accepting them based on the probability density
	\item Repeating until the desired number of samples is obtained
\end{enumerate}

\section{Parallel Processing of Parameter Combinations}
\label{sec:parallel_processing}

\noindent
The computationally intensive nature of 2DES simulations, particularly when averaging over frequency samples and phase combinations, necessitates efficient parallel processing. The averaged response is calculated as:

\begin{equation}
	\label{eq:averaged_response}
	\langle P^{(3)}(t_{\text{coh}}, T_{\text{wait}}, t_{\text{det}}) \rangle = \frac{1}{N_\omega N_\phi} \sum_{i=1}^{N_\omega} \sum_{j=1}^{N_\phi} P^{(3)}_{\omega_i, \phi_j}(t_{\text{coh}}, T_{\text{wait}}, t_{\text{det}})
\end{equation}

\noindent
where $N_\omega$ represents the number of frequency samples and $N_\phi$ the number of phase combinations. The implementation utilizes Python's \texttt{ProcessPoolExecutor} to distribute calculations across available CPU cores, significantly reducing computation time for parameter sweeps.

\section{Fourier Transformation and Spectral Analysis}
\label{sec:fourier_transformation}

The conversion from time-domain to frequency-domain spectra utilizes two-dimensional Fourier transformation:

\begin{equation}
	\label{eq:2d_fourier_transform}
	S(\omega_{\text{coh}}, \omega_{\text{coh}}) = \int_{-\infty}^{\infty} \int_{-\infty}^{\infty} E_{\vec{k}_{S}}^{(3)}(t_{\text{coh}}, T, t) e^{-i\omega_{\text{coh}} t_{\text{coh}}} e^{-i\omega_{\text{coh}} t_{\text{coh}}} \, dt_{\text{coh}} \, dt
\end{equation}

\noindent
In the discrete implementation, this becomes a Fast Fourier Transform (FFT) operation:

\begin{equation}
	\label{eq:discrete_2d_fft}
	S[k,l] = \sum_{m=0}^{M-1} \sum_{n=0}^{N-1} P^{(3)}[m,n] \exp\left(-i\frac{2\pi km}{M}\right) \exp\left(-i\frac{2\pi ln}{N}\right)
\end{equation}

\noindent
The frequency axes are converted to wavenumber units ($10^4$ cm$^{-1}$) using:

\begin{equation}
	\label{eq:frequency_to_wavenumber}
	\nu = \frac{f}{c} \times 10^4 \text{ cm}^{-1}
\end{equation}

\noindent
where $f$ is the frequency in cycles/fs and $c = 2.998 \times 10^{-5}$ cm/fs is the speed of light.

\section{Data Processing and Visualization}
\label{sec:data_visualization}

\noindent
The 2D spectra visualization system represents different aspects of the complex-valued data:

\begin{itemize}
	\item Real part: Absorptive component containing peak positions
	\item Imaginary part: Dispersive component providing line shape information
	\item Absolute value: Overall signal strength
	\item Phase: Phase relationship between excitation and detection processes
\end{itemize}

\noindent
For data containing both positive and negative values, a custom white-centered colormap enhances visualization of signal features. The normalization scheme:

\begin{equation}
	\label{eq:data_normalization}
	\text{data}_{\text{normalized}} = \frac{\text{data}}{\max(|\text{data}|)}
\end{equation}

\noindent
ensures consistent visualization across different parameter sets while preserving relative signal amplitudes.

\section{Extending Time and Frequency Axes}
\label{sec:extending_axes}

\noindent
To improve Fourier transform resolution and reduce artifacts from finite sampling, the implementation supports zero-padding of time-domain data:

\begin{equation}
	\label{eq:zero_padding_matrix}
	\text{data}_{\text{extended}} = \begin{bmatrix}
		0 & \cdots & 0 & \text{data} & 0 & \cdots & 0
	\end{bmatrix}
\end{equation}

\noindent
The corresponding time axes extend proportionally:

\begin{equation}
	\label{eq:extended_time_axis}
	t_{\text{extended}} = \{t_0 - n_{\text{pre}} \Delta t, \ldots, t_0, \ldots, t_N, \ldots, t_N + n_{\text{post}} \Delta t\}
\end{equation}

\noindent
where $n_{\text{pre}}$ and $n_{\text{post}}$ represent the number of zero-padded points before and after the original data.

\section{Global Data Combination for Multiple Waiting Times}
\label{sec:global_data_combination}

\noindent
For studies of temporal evolution in coherence phenomena, the simulation supports scanning over multiple waiting times. The local time and frequency data are mapped to global axes through:

\begin{equation}
	\label{eq:global_mapping_corrected}
	\mathrm{global\_data}[\mathrm{idx}_{t_{\mathrm{coh}}}, \mathrm{idx}_{t}] += \mathrm{local\_data}[\mathrm{local\_idx}_{t_{\mathrm{coh}}}, \mathrm{local\_idx}_{t}]
\end{equation}

\noindent
where $\text{idx}_{t_{\text{coh}}}$ and $\text{idx}_t$ represent indices in the global arrays corresponding to local values of $t_{\text{coh}}$ and $t_{\text{det}}$. The global data normalization:
\begin{equation}
	\label{eq:global_normalization}
	\text{global\_data}_{\text{normalized}} = \frac{\text{global\_data}}{N_{T_{\text{wait}}}}
\end{equation}

\noindent
produces the average spectrum over all waiting times, enabling analysis of relaxation and dephasing processes.

\section{Connection to Theoretical Framework}
\label{sec:connection_to_theory}

\noindent
The numerical implementation directly implements the theoretical concepts established in previous chapters:

\begin{itemize}
	\item Quantum system evolution follows the Redfield master equation for open quantum systems (Chapter~\ref{chapter_open_quantum_systems})
	\item Polarization calculations implement the nonperturbative approach for 2DES theory
	\item Spectral analysis methods extract signatures of quantum coherence in biological systems
	\item Environmental coupling effects are incorporated through systematic parameter studies
\end{itemize}

\noindent
This computational framework provides a robust platform for simulating 2DES spectra of complex molecular systems and extracting quantum dynamical parameters relevant to biological function.

\section{Validation and Performance Considerations}
\label{sec:validation_performance}

\noindent
The implementation includes validation procedures comparing simulation results against analytical solutions for simple systems. Performance optimization through vectorized operations and parallel processing enables parameter sweeps necessary for comprehensive system characterization.

\noindent
Special considerations for biological systems include:

\begin{enumerate}
	\item Larger inhomogeneous broadening due to complex environments
	\item Faster dephasing from environmental coupling
	\item Multiple chromophore contributions to the signal
	\item Energy transfer processes on various timescales
\end{enumerate}

\noindent
The simulation framework accommodates these aspects through customizable system parameters and flexible averaging procedures, providing a versatile tool for investigating quantum coherence phenomena in biological systems.