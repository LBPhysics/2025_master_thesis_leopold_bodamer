% Chapter Template

\chapter{Rabi Oscillations and Related Concepts} % Main chapter title

\label{chapter_rabi_oscillations} % For referencing this chapter elsewhere, use \ref{chapter_rabi_oscillations}

%-----------------------------------
%	SECTION: Introduction
%-----------------------------------
\section{Introduction}
\subsection{Density Matrix Formalism}

The density matrix formalism provides a powerful framework to describe the dynamics of quantum systems, especially when dealing with mixed states or decoherence.
The density matrix \(\rho\) is defined as:
\begin{equation}
	\rho = |\psi\rangle \langle \psi|,
	\label{eq:DensityMatrix}
\end{equation}
for pure states, and as a statistical mixture for mixed states. \todoidea{add some more definitions like in the paper \cite{grolletal2025fundamentalsheterodynewave} p 16,17}
In this formalism, the diagonal elements represent populations and the off-diagonal elements represent coherences between states.
Coherences are phase relations between different quantum states, which are crucial for interference.
For example for a two level system a clear phase relation and a pure quantum state would have off diagonal elements of $ \rho_{ij} = 1/2$.
This state is often referred to as a coherent superposition of the two states.

When coupling a system to an environment, the environment is responsible for decoherence.
The state evolves over time to a purely statistical mixture of states.

The time evolution of the density matrix is governed by the Liouville-von Neumann equation:
\begin{equation}
	\frac{\partial \rho}{\partial t} = -\frac{i}{\hbar} [H, \rho],
	\label{eq:Liouville}
\end{equation}
where \(H\) is the system Hamiltonian.

In the presence of decoherence or dissipation, the dynamics can be described using the Lindblad master equation:
\begin{equation}
	\frac{\partial \rho}{\partial t} = -\frac{i}{\hbar} [H, \rho] + \sum_k \mathcal{L}_k(\rho),
	\label{eq:Lindblad}
\end{equation}
where \(\mathcal{L}_k(\rho)\) are Lindblad operators modeling the interaction with the environment.

%----------------------------------------------------------------------------------------
%	SECTION: Two-Level Atom Master Equation in the Rotating Frame with RWA
%----------------------------------------------------------------------------------------

\section{Two-Level Atom Master Equation in the Rotating Frame with RWA}

Rabi oscillations describe the coherent oscillatory behavior of a two-level quantum system interacting with a resonant electromagnetic field.
This phenomenon is fundamental in quantum mechanics and quantum optics, with applications in quantum computing, spectroscopy, and atomic physics.

\subsection{Hamiltonian in the Laboratory Frame}

Consider a two-level system with states \(|g\rangle\) (ground state) and \(|e\rangle\) (excited state).
The energy separation between the two states is given by:
\begin{equation}
	\omega_0 = \frac{E_e - E_g}{\hbar},
	\label{eq:EnergySeparation}
\end{equation}
where \(E_e\) and \(E_g\) are the energies of the excited and ground states, respectively.

\subsubsection{Atomic Hamiltonian}
The free atomic Hamiltonian is expressed as:
\begin{equation}
	H_0 = \frac{\hbar\omega_0}{2} \sigma_z, \qquad \sigma_z = \ket{e}\bra{e} - \ket{g}\bra{g}
	\label{eq:AtomicHamiltonian}
\end{equation}

\subsubsection{Electric Field}
The electric field is given by:
\begin{equation}
	\mathcal{E}(t) = E_0 e^{i\phi} e^{-i\omega_L t}, \qquad
	E_{\text{phys}}(t) = \mathcal{E}(t) + \mathcal{E}^*(t)
	\label{eq:ElectricFieldComplex}
\end{equation}
where \(E_0\) is the field amplitude, \(\phi\) is the phase, and \(\omega_L\) is the laser frequency.

\subsubsection{Interaction Hamiltonian}
Using the electric dipole interaction with raising and lowering operators \(\sigma_\pm = (\sigma_x \pm i\sigma_y)/2\):
\begin{equation}
	H_{\text{int}}(t) = -\mu \left( \mathcal{E}(t) \sigma_+ + \mathcal{E}^*(t) \sigma_- \right)
	\label{eq:InteractionHamiltonianDipole}
\end{equation}
where \(\mu\) is the dipole matrix element.

\subsubsection{Total Hamiltonian}
The total Hamiltonian of the system is:
\begin{equation}
	H(t) = H_0 + H_{\text{int}}(t)
	\label{eq:TotalHamiltonian}
\end{equation}

The time evolution of the system is governed by the time-dependent Schrödinger equation \eqref{eq:SchrödingerEquation}

\subsection{Transform to the Rotating Frame}

Define the unitary transformation:
\begin{equation}
	U(t) = \exp\left(i \omega_L t \frac{\sigma_z}{2}\right)
	\label{eq:UnitaryRotatingFrame}
\end{equation}

Transform operators as \(\tilde{X} = U X U^\dagger\), and the density matrix as \(\tilde{\rho} = U \rho U^\dagger\).

Useful identities for the transformation:
\begin{equation}
	U \sigma_\pm U^\dagger = e^{\pm i \omega_L t} \sigma_\pm, \qquad
	U \sigma_z U^\dagger = \sigma_z
	\label{eq:TransformationIdentities}
\end{equation}

\subsubsection{Transformed Hamiltonian Terms}

\begin{itemize}
	\item \textbf{Free Hamiltonian:}
	      \begin{equation}
		      \tilde{H}_0 = \frac{\hbar(\omega_0 - \omega_L)}{2} \sigma_z = \frac{\hbar\Delta}{2} \sigma_z
		      \label{eq:TransformedFreeHamiltonian}
	      \end{equation}
	      where \(\Delta = \omega_0 - \omega_L\) is the detuning.

	\item \textbf{Transformed interaction Hamiltonian:}
	      \begin{equation}
		      \tilde{H}_{\text{int}} = -\mu \left( E_0 e^{i\phi} \sigma_+ + E_0 e^{-i\phi} \sigma_- \right)
		      \label{eq:TransformedInteractionHamiltonian}
	      \end{equation}

	\item Defining the Rabi frequency \(\Omega = \mu E_0 / \hbar\), the total rotating-frame Hamiltonian becomes:
	      \begin{equation}
		      \boxed{
			      \tilde{H} = \frac{\hbar\Delta}{2} \sigma_z - \hbar \Omega \left( e^{i\phi} \sigma_+ + e^{-i\phi} \sigma_- \right)
		      }
		      \label{eq:RotatingFrameHamiltonian}
	      \end{equation}
\end{itemize}

\subsection{Master Equation in the Rotating Frame (with RWA)}

The master equation in the rotating frame is:
\begin{equation}
	\dot{\rho} = -i[\tilde{H}, \rho] + \mathcal{D}(\rho)
	\label{eq:MasterEquationRotatingFrame}
\end{equation}

The dissipator includes spontaneous emission and pure dephasing:
\begin{equation}
	\mathcal{D}(\rho) =
	\gamma_0 \left( \sigma_- \rho \sigma_+ - \frac{1}{2} \{ \sigma_+ \sigma_-, \rho \} \right)
	+ \frac{\gamma_\varphi}{2} \left( \sigma_z \rho \sigma_z - \rho \right)
	\label{eq:DissipatorOperator}
\end{equation}

Define the total decoherence rate:
\begin{equation}
	\Gamma = \frac{\gamma_0}{2} + \gamma_\varphi
	\label{eq:TotalDecoherenceRate}
\end{equation}

\subsection{Equations of Motion}

Write the density matrix as:
\begin{equation}
	\rho = \begin{pmatrix}
		\rho_{gg} & \rho_{ge} \\
		\rho_{eg} & \rho_{ee}
	\end{pmatrix}, \qquad \rho_{ge} = \rho_{eg}^*
	\label{eq:DensityMatrixElements}
\end{equation}

Let \(\Omega_c = \Omega e^{i\phi}\). The equations of motion are:
\begin{equation}
	\boxed{
		\begin{aligned}
			\dot{\rho}_{ee} & = i \Omega_c \rho_{ge} - i \Omega_c^* \rho_{eg} - \gamma_0 \rho_{ee}            \\[6pt]
			\dot{\rho}_{gg} & = -i \Omega_c \rho_{ge} + i \Omega_c^* \rho_{eg} + \gamma_0 \rho_{ee}           \\[6pt]
			\dot{\rho}_{eg} & = i \Delta \rho_{eg} + i \Omega_c (\rho_{gg} - \rho_{ee}) - \Gamma \rho_{eg}    \\[6pt]
			\dot{\rho}_{ge} & = -i \Delta \rho_{ge} - i \Omega_c^* (\rho_{gg} - \rho_{ee}) - \Gamma \rho_{ge}
		\end{aligned}
	}
	\label{eq:BlochEquations}
\end{equation}

On resonance (\(\Delta = 0\)), these reduce to the form implemented in the Liouvillian matrix in numerical calculations.

\subsection{Liouvillian Matrix Form}

Flatten the density matrix into a column vector:
\todoidea{
	use the column stacking version like qutip uses!
}
\begin{equation}
	\vec{\rho} =
	\begin{pmatrix}
		\rho_{gg} \\
		\rho_{ge} \\
		\rho_{eg} \\
		\rho_{ee}
	\end{pmatrix}
	\label{eq:DensityMatrixVector}
\end{equation}

Then the master equation becomes:
\begin{equation}
	\dot{\vec{\rho}} = L(t) \vec{\rho}
	\label{eq:LiouvillianMatrixForm}
\end{equation}

Where \(L(t)\) is the time-dependent Liouvillian matrix, using:
\begin{equation}
	\Omega_c(t) = \frac{\mu}{\hbar} E(t), \quad \text{with } E(t) = E_0 e^{i\phi}
	\label{eq:TimeDepRabiFreq}
\end{equation}

This matrix form is equivalent to the operator master equation above, and is suitable for numerical integration.

\subsection{Rabi Frequency and Dynamics}

The Rabi frequency \(\Omega_R\) determines the oscillation rate between the two levels:
\begin{equation}
	\Omega_R = \sqrt{\Delta^2 + \Omega^2}
	\label{eq:RabiFrequency}
\end{equation}

The phase \(\phi\) does not change the \textbf{Rabi frequency}, but it \textbf{rotates the axis of Rabi oscillations} in the Bloch sphere — i.e., it changes the \textbf{initial direction} of the drive.

In matrix form, in the basis \(\{ \ket{e}, \ket{g} \}\), the RWA Hamiltonian is:
\begin{equation}
	\tilde{H}_{\text{RWA}} = \frac{\hbar}{2}
	\begin{pmatrix}
		\Delta             & -\Omega e^{i\phi} \\
		-\Omega e^{-i\phi} & -\Delta
	\end{pmatrix}
	\label{eq:RWAHamiltonianMatrix}
\end{equation}

\subsection{Recovery of Laboratory Frame Density Matrix}

To recover the entries of the original density matrix \(\rho(t)\) from the evolved density matrix in the rotating frame \(\tilde{\rho}(t)\), we use the inverse transformation:
\begin{equation}
	\rho(t) = U^\dagger(t) \tilde{\rho}(t) U(t)
	\label{eq:RecoverOriginalDensityMatrix}
\end{equation}

The entries of the original density matrix \(\rho(t)\) are related to the entries of the density matrix in the rotating frame \(\tilde{\rho}(t)\) by:
\begin{align}
	\rho_{gg}(t) & = \tilde{\rho}_{gg}(t) \label{eq:RecoveryGG}                  \\
	\rho_{ee}(t) & = \tilde{\rho}_{ee}(t) \label{eq:RecoveryEE}                  \\
	\rho_{ge}(t) & = e^{i\omega_L t} \tilde{\rho}_{ge}(t) \label{eq:RecoveryGE}  \\
	\rho_{eg}(t) & = e^{-i\omega_L t} \tilde{\rho}_{eg}(t) \label{eq:RecoveryEG}
\end{align}

%----------------------------------------------------------------------------------------
%	SECTION: Applications and Implications
%----------------------------------------------------------------------------------------

\section{Applications and Implications}

Rabi oscillations and the associated theoretical tools, such as the RWA and density matrix formalism, have wide-ranging applications:
\begin{itemize}
	\item \textbf{Quantum Computing:} Rabi oscillations are used to implement quantum gates by precisely controlling the population of qubits.
	\item \textbf{Spectroscopy:} The Rabi frequency provides information about the interaction strength between light and matter.
	\item \textbf{Atomic Physics:} Understanding Rabi oscillations is essential for manipulating atomic states in experiments.
\end{itemize}

These concepts form the foundation for advanced topics in quantum mechanics and quantum technologies.
