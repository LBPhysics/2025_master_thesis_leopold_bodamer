\chapter{Introduction} % Main chapter title
\label{Chapter1} % Change X to a consecutive number; for referencing this chapter elsewhere, use \ref{ChapterX}

%----------------------------------------------------------------------------------------
%	SECTION 1
%----------------------------------------------------------------------------------------
%----------------------------------------------------------------------------------------
%	SECTION 1: Coherence and Excitation Transport
%----------------------------------------------------------------------------------------

\section{Coherence and Excitation Transport}

In this chapter, we aim to explain the phenomena of long coherences (lifetimes) and the excitation transport of light on a microtubule. The proposed model takes the following approach:

\begin{itemize}
    \item The microtubule is modeled as a cylindrical structure consisting of nodes. Each node represents an atom, which is modeled as a two-level system. The number of atoms, \( N_{\text{atoms}} \), is determined by the number of chains (\( n_{\text{chains}} \)) and the number of rings (\( n_{\text{rings}} \)), assuming fixed positions for these nodes.
    \item The system is restricted to a single excitation.
    \item A time-dependent coupling to an electric field is proposed, which may be either classical or quantum in nature. This coupling is intended to facilitate spectroscopy.
    \item Two types of Lindblad operators are introduced to model dissipation processes. Specifically:
    \begin{enumerate}
        \item Spontaneous decay
        \item Dephasing
    \end{enumerate}
\end{itemize}
The Lindblad operators introduced to model the spontaneous decay and dephasing processes for each individual atom are defined as follows:

\begin{align}
    C_{\text{decay}}^{(i)} &= \sqrt{\gamma_0} \, \sigma_-^{(i)}, \\
    C_{\text{dephase}}^{(i)} &= \sqrt{\gamma_\phi} \, \sigma_z^{(i)},
\end{align}

where:
\begin{itemize}
    \item \( C_{\text{decay}}^{(i)} \) describes the spontaneous decay of the \(i\)-th atom, with a rate given by \(\gamma_0\).
    \item \( C_{\text{dephase}}^{(i)} \) describes the dephasing of the \(i\)-th atom, with a rate given by \(\gamma_\phi\).
    \item \( \sigma_-^{(i)} \) is the lowering operator for the \(i\)-th atom, and \( \sigma_z^{(i)} \) is the Pauli \( z \)-operator for the \(i\)-th atom.
\end{itemize}



\newpage











\section{Motivation}
\noindent

%----------------------------------------------------------------------------------------
%	SECTION 1
%----------------------------------------------------------------------------------------
%\section{Objective}
%\vspace{0.5cm}
%\noindent
%The goal of this thesis is
%to perform robust directional photon routing on atomic systems in free-space using subradiant states.
%Focusing on a Y-shaped atomic tree, different topologies are explored to enable long-lived information transport as a proof of concept.
%
%\section{Outline}
%This thesis is structured as follows.
%Chapter \ref{Chapter2} introduces the theoretical background.
%It covers the concepts of open quantum systems,
%subradiance and superradiance, the Green tensor, and the reciprocal space.
%These tools are essential foundations for describing atom-atom interactions in free space,
%including dipole-dipole interactions and coupling to a photonic bath.
%%After this chapter, the reader already knows...
%The quantum router of \cite{Startingpoint} is presented and summarized in Chapter \ref{Chapter3}.
%It introduces the concepts of graph theory and explains how quantum evolution on a graph topology can be utilized to achieve directional routing of information.
%Chapter \ref{Chapter4} will be the core of this thesis, adapting this model to an atomic system.
%This chapter delves into the challenges of implementing directional routing in a fully connected atomic system and investigates various solutions to control the phase of interactions.
%It further extends the analysis to systems with a larger number of atoms, focusing on coupling control and routing capabilities in different configurations, such as equilateral and isosceles triangles.
%Chapter \ref{Chapter5} concludes the thesis by summarizing the results and discussing potential future directions in the field of quantum routing in atomic systems.


It is widely assumed that one of the crucial tasks currently facing quantum theorists
is to understand and characterize the behaviour of realistic quantum systems. In
any experiment, a quantum system is subject to noise and decoherence due to the
unavoidable interaction with its surroundings. The theory of open quantum systems
aims at developing a general framework to analyze the dynamical behaviour of systems
that, as a result of their coupling with environmental degrees of freedom, will no
longer evolve unitarily. \cite{rivas_markovian_2010}
\newline
2DES> \cite{krumland_two-dimensional_2023}, \cite{segarra-marti_towards_2018}, \cite{sun_two-dimensional_2024}
\newline
NONlinear Optics> \cite{hamm_principles_nodate}, \cite{mukamel_principles_1995}
\newline
Spectroscopy investigates the interaction between matter and electromagnetic radiation, offering a means to analyze composition and structure.
Central to this analysis is the understanding of how molecules respond to specific frequencies of light, revealing information about their energy levels and bonding.
Key concepts include wavelength ($\lambda$), wavenumber ($\bar{\nu}$), and frequency ($\nu$).
Wavelength, the distance between successive wave crests, is typically measured in nanometers or micrometers.
Wavenumber, expressed in inverse centimeters (cm$^{-1}$), represents the number of waves per unit distance and is directly proportional to energy, defined as $\bar{\nu} = 1/\lambda$ (where $\lambda$ is in cm).
Frequency, the number of wave cycles per second, is measured in Hertz (Hz), and the angular frequency ($\omega$) is related to frequency by $\omega = 2\pi\nu$.
The relationship between angular frequency and wavenumber is given by $\omega = 2\pi c \bar{\nu}$, where $c$ is the speed of light.

\newline
Then I turned everything into $fs^{-1}$.


\newline
Spectrometers are instruments designed to measure the intensity of light as a function of wavelength or frequency.
Different types of spectrometers are employed for various regions of the electromagnetic spectrum.
Notably, UV-Vis spectrometers analyze absorption and transmission of ultraviolet and visible light, while infrared (IR) spectrometers measure the absorption of infrared light, providing insights into molecular vibrations.
Nuclear Magnetic Resonance (NMR) spectrometers probe the magnetic properties of atomic nuclei, revealing molecular structure.




\newpage
\section{Bath Correlators and Transition Rates}
The quantum mechanical state of a bosonic bath in thermal equilibrium at temperature \( T \) is given by the density matrix:
\[
\rho = \frac{1}{Z} e^{-\beta H}, \quad Z = \text{Tr}\left[e^{-\beta H}\right]
\]
And the expectation value of an operator \( A \) in a system \( S \) is given by:
\[
\langle A \rangle_T = \text{Tr_S}[\rho_S A] = \frac{1}{Z} \sum_n e^{-\beta \hbar \omega_n} \langle n | A | n \rangle,
\]

where \(|n\rangle\) are number states. 
The inverse temperature \(\beta\) is defined as:

\[
\beta = \frac{1}{k_B T}
\]

\subsection{Bath Correlator}

The bath correlator is given by:
\[
C(\tau) = \langle B(\tau) B(0) \rangle
\]
where the bath operator \( B \) is defined as:
\[
B = \sum_{n=1}^{\infty} c_n x_n
\]
and the bath operator in terms of creation and annihilation operators is:
\[
B(0) = \sum_{n=1}^{\infty} c_n \sqrt{\frac{1}{2 m_n \omega_n}} (b_n + b_n^\dagger),
\]
\[
B(\tau) = \sum_{n=1}^{\infty} c_n \sqrt{\frac{1}{2 m_n \omega_n}} \left( b_n e^{-i \omega_n \tau} + b_n^\dagger e^{i \omega_n \tau} \right).
\]
Substituting \( B(\tau) \) and \( B(0) \) into the bath correlator expression:
\[
C(\tau) = \left\langle \sum_{n=1}^{\infty} c_n \sqrt{\frac{1}{2 m_n \omega_n}} (b_n e^{-i \omega_n \tau} + b_n^\dagger e^{i \omega_n \tau}) \sum_{m=1}^{\infty} c_m \sqrt{\frac{1}{2 m_m \omega_m}} (b_m + b_m^\dagger) \right\rangle.
\]
Applying the thermal expectation values:
\[
\langle b_n b_m^\dagger \rangle = \delta_{nm} (n_n + 1), \quad \langle b_n^\dagger b_m \rangle = \delta_{nm} n_n,
\]
where \( n_n \) is the Bose-Einstein distribution:
\[
n_n = \frac{1}{e^{\beta \omega_n} - 1}.
\]
We get:
\[
C(\tau) = \sum_{n=1}^{\infty} \frac{c_n^2}{2 m_n \omega_n} \left[ (n_n + 1) e^{-i \omega_n \tau} + n_n e^{i \omega_n \tau} \right].
\]
Next, we express the correlation in terms of the spectral density \( J(\omega) \), which is defined as:
\[
J(\omega) = \pi \sum_{n=1}^{\infty} \frac{c_n^2}{2 m_n \omega_n} \delta(\omega - \omega_n).
\]
Thus, the bath correlator becomes:
\[
C(\tau) = \int_0^\infty d\omega \frac{J(\omega)}{\pi} \left[ (n(\omega) + 1) e^{-i \omega \tau} + n(\omega) e^{i \omega \tau} \right].
\]
Rearranging the terms, we get:
\[
C(\tau) = \int_0^\infty d\omega \frac{J(\omega)}{\pi} \left[ \coth\left( \frac{\beta \omega}{2} \right) \cos(\omega \tau) - i \sin(\omega \tau) \right].
\]
This is the desired result:
\[
C(\tau) = \int_0^\infty d\omega \frac{J(\omega)}{\pi} \left( \coth\left( \frac{\omega}{2} \right) \cos(\omega \tau) - i \sin(\omega \tau) \right)
\]




\newpage
\section*{Conversion from Discrete Sum to Continuous Spectral Density}

\noindent In the theory of open quantum systems, one often moves from a discrete description of the bath to a continuous one. This is summarized by the transformation:
\[
\sum_{j=1}^{M} g_j^2 \,\delta(\omega - \omega_j)
\;\;\longrightarrow\;\;
J(\omega).
\]

\subsection*{1. Discrete Modes}
The bath is initially described as a set of $M$ harmonic oscillators:
\begin{itemize}
  \item Each oscillator has frequency $\omega_j$.
  \item Each oscillator couples to the system with coupling constant $g_j$.
\end{itemize}
The spectral contribution of each oscillator is:
\[
\sum_{j=1}^{M} g_j^2\,\delta(\omega - \omega_j).
\]

\subsection*{2. Density of States}
As $M \to \infty$ and the frequencies $\{\omega_j\}$ become dense, replace the sum by an integral:
\[
\sum_{j=1}^{M} \quad \longrightarrow \quad \int d\omega'\,\rho(\omega'),
\]
where $\rho(\omega')$ is the density of states, indicating how many modes lie near frequency $\omega'$.

\subsection*{3. Frequency-Dependent Coupling}
In the continuum limit, the coupling $g_j$ becomes a function of frequency, $g(\omega')$. Hence:
\[
g_j^2 \;\;\longrightarrow\;\; g(\omega')^2.
\]

\subsection*{4. Form of $J(\omega)$}
Putting these together, one obtains
\[
\sum_{j=1}^{M} g_j^2\,\delta(\omega - \omega_j)
\;\;\longrightarrow\;\;
\int_{0}^{\infty} d\omega'\,\rho(\omega')\,g(\omega')^2\,\delta(\omega - \omega').
\]
Using the sifting property of the delta function, this becomes
\[
J(\omega) \;=\; \rho(\omega)\,g(\omega)^2.
\]

\subsection*{5. Ohmic Spectral Density with Exponential Cutoff}
Source: Ulrich Weiss chapter 7.3:
(*my interpretation: The Redfield equation is influenced by classical phenomenological models of dissipation*)
In the strict Ohmic case, damping is frequency-independent. $ \gamma(\omega) = \gamma $. 
In this case the spectral density is given by::
\begin{equation}
  J(\omega) \propto \gamma \omega
\end{equation}
for all frequencies $\omega$, which implies memoryless frinction. 
However, in reality this idealization can't hold because every spectral density falls to 0 fro $\omega \to \infty$.
Analogy to classical resistor, where the dissipation is proportional to the current.

\newline
A commonly used form in open quantum system models is the Ohmic spectral density with an exponential cutoff:
\[
J(\omega) \;=\; \alpha \,\omega\, e^{-\omega/\omega_c},
\]
where:
\begin{itemize}
  \item $\alpha$ is a dimensionless coupling constant,
  \item $\omega_c$ is a high frequency cutoff.
\end{itemize}
``Ohmic'' means $J(\omega) \propto \omega$ for small $\omega$, and the exponential cutoff ensures convergence at large $\omega$.