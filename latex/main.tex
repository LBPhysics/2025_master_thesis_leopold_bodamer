\documentclass[
11pt, english, 
draft, % no pictures, links, overfull hboxes indicated
toctotoc, 		      % add table of contents to the table of contents
%liststotoc, 	       % add the list of figs/tables/etc to the table of contents
%nolistspacing, 	% If the document is onehalfspacing or doublespacing, uncomment this to set spacing in lists to single
%parskip, 			% add space between paragraphs
%headsepline, 		% get a line under the header
%chapterinoneline, 	% to place the chapter title next to the number on one line
%consistentlayout, 	% to change the layout of the declaration, abstract and acknowledgements pages to match the default layout
]{style_thesis}

%-------------------------------------------------------------------------------
%   Define the LIBRARY and other packages
%-------------------------------------------------------------------------------
% actually import stzle with style=customstyle???? -> not needed
\usepackage[
    backend=biber,
    style=phys,         % The biblatex-phys style for physics journals
    articletitle=true,
    biblabel=brackets,  % Puts numbers in brackets, e.g., [1]
    citestyle=numeric-comp % Compresses citations, e.g., [1, 2, 3] -> [1-3]
]{biblatex}%\renewcommand*{\bibfont}{\small} % Optional: Make the bibliography font size smaller
\addbibresource{bib/my_bibliography.bib}

% Added content-related packages and definitions
\usepackage{graphicx} % Required to include images
\graphicspath{/figures}
\usepackage{amsmath, amssymb, amsfonts} % Math packages
\usepackage{dsfont} % gives \mathds{1}
\usepackage{braket} % For quantum mechanics notation
\usepackage{fontspec} % For loading fonts in LuaLaTeX/XeLaTeX
%\usepackage{mathpazo} % TODO Use the Palatino font by default, % ONLY WORKS WITH OLD PDFLATEX VERSION? 
\newcommand{\todoimp}[1]{\todo[inline,color=red!70]{\textbf{IMPORTANT:} #1}}
\newcommand{\todoidea}[1]{\todo[inline,color=green!60]{\textbf{IDEA:} #1}}
\newcommand{\todoeq}[1]{\todo[inline,color=blue!50]{\textbf{EQUATION:} #1}}
\newcommand{\todoref}[1]{\todo[inline,color=purple!50]{\textbf{REFERENCE:} #1}}
\newcommand{\todofix}[1]{\todo[inline,color=orange!70]{\textbf{FIX:} #1}}

%-------------------------------------------------------------------------------
%	THESIS INFORMATION
%-------------------------------------------------------------------------------

\thesistitle{Master thesis}
\supervisor{\href{https://ic1.ugr.es/members/dmanzano/home/}{Prof. Dr. Daniel \textsc{Manzano Diosdado}}}
\examiner{\href{https://sites.google.com/site/beatrizolmosphysics/}{Prof. Dr. Beatriz \textsc{Olmos Sanchez}}}
\degree{Master of Science}
\author{Leopold \textsc{Bodamer}}
\addresses{Wolfsbergallee 17b, 75177 Pforzheim}
\subject{Theoretical Atomic Physics and Synthetic Quantum Systems}
\universityone{\href{https://uni-tuebingen.de\\}{Eberhard Karls Universität Tübingen}}

\department{\href{https://uni-tuebingen.de/fakultaeten/mathematisch-naturwissenschaftliche-fakultaet/fachbereiche/physik/institute/institut-fuer-theoretische-physik/arbeitsgruppen/}{Institut für Theoretische Physik}}
\group{\href{https://uni-tuebingen.de/fakultaeten/mathematisch-naturwissenschaftliche-fakultaet/fachbereiche/physik/institute/institut-fuer-theoretische-physik/arbeitsgruppen/ag-lesanovsky/}{Theoretical Atomic Physics and Synthetic Quantum Systems}}
\faculty{\href{https://uni-tuebingen.de/fakultaeten/mathematisch-naturwissenschaftliche-fakultaet/fakultaet/}{Mathematisch-Naturwissenschaftliche Fakultät}}

\departmenttwo{\href{https://www.ugr.es/en/about/organization/entities/department-electromagnetism-and-matter-physics}{Department of Electromagnetism and Matter Physics}}
\grouptwo{\href{https://ic1.ugr.es/members/qtc/}{Quantum Thermodynamics and Computation Group}}
\facultytwo{\href{https://fciencias.ugr.es/}{Faculty of Sciences}}
\universitytwo{\href{https://www.ugr.es\\}{University of Granada}}
% NOTE could also add a faculty / adress of the universities

%\includeonly{chapters/c10_introduction} % Place this BEFORE \begin{document} to only compile specific chapter

\begin{document}
%\ The width of this document is: \the\textwidth

\frontmatter
\pagestyle{plain}

\input{chapters/c00_titlepage}
\input{chapters/c01_declaration}
\input{chapters/c02_acknow_lists_abbrs_consts_symbls_dedication}
%-------------------------------------------------------------------------------
%	ABSTRACT PAGE
%-------------------------------------------------------------------------------
\begin{abstract}
	\addchaptertocentry{\abstractname}
	todoidea{This thesis investigates the application of two-dimensional photon-echo spectroscopy to model quantum systems, with a focus on open quantum system dynamics. We develop a comprehensive theoretical framework that combines the response-function formalism !NON-perturbatively! of nonlinear spectroscopy with master-equation approaches for open quantum systems (Qutip). Starting from minimal reference models, including a single qubit and a four-level system formed by two coupled qubits, we validate our methods and apply the model to a system of $N$ qubits arranged on a cylindrical geometry, inspired by the structural motifs of microtubules. Our results demonstrate the capability of two-dimensional spectroscopy to probe coherence and energy transfer dynamics in these systems, providing insights into their quantum behavior and interactions with the environment. This work lays the groundwork for future experimental and theoretical studies in the field of quantum biology and quantum information science.}
\end{abstract}

%-------------------------------------------------------------------------------
%	CONTENTS
%-------------------------------------------------------------------------------

\mainmatter
\pagestyle{thesis}

%\include{chapters/_template_chapter}
%\include{chapters/c10_introduction}
% !TEX root = ../main.tex
\chapter{Open Quantum Systems} % Main chapter title
\label{chapter_open_quantum_systems} % Label for referencing this chapter

%------------------------------------------------------------------------------
%	SECTION 1: Introduction to Open Quantum Systems / Conceptual background
%------------------------------------------------------------------------------

\section{Conceptual background}
\label{sec:conceptual_background}

Any real-world quantum system is not perfectly isolated. Instead, it interacts with its surrounding environment. This also happens in the most shielded experiments, at least to some degree. Unlike theoretical closed quantum systems that evolve unitarily according to the well-known Schrödinger equation 

\begin{equation}
	i\hbar \frac{\partial}{\partial t} |\psi(t)\rangle = H(t) |\psi(t)\rangle ,
	\label{eq:SchrödingerEquation}
\end{equation}

\noindent
open quantum systems experience non-unitary evolution due to their coupling with an external environment or reservoir.
Whenever the system is driven by external perturbations, this coupling makes the system relax back to the equilibrium defined by this environment~\cite{breuerpetruccione2009theoryopenquantum, weiss2012quantumdissipativesystems}.
Open quantum systems theory emerges from this realization, that perfect isolation of a quantum system is practically impossible. For instance, atoms are subject to electromagnetic field fluctuations (vacuum fluctuations)~\cite{breuerpetruccione2009theoryopenquantum}, quantum dots in solid-state environments couple to phonon baths~\cite{weiss2012quantumdissipativesystems}, and molecular systems interact with surrounding solvent molecules~\cite{mukamel1995principlesnonlinearoptical}. Quantum computers are particularly sensitive to environmental noise, which can rapidly destroy quantum coherence~\cite{laddetal2010quantumcomputers}, while biological quantum systems such as photosynthetic complexes operate in inherently noisy cellular environments. \todoref{schlosshauer2007decoherencebook}. 
These diverse scenarios all require a theoretical framework that accounts for the coexistence of quantum effects and environmental influences.


\subsection{Phenomena: Decoherence and Dissipation}
\label{subsec:phenomena_decoherence_dissipation}

\noindent
The interaction with the environment leads to several fundamental phenomena that distinguish open systems from isolated, closed quantum systems.

\noindent
A central effect is \textbf{decoherence}, where quantum superposition states lose their phase relationships due to entanglement with the environment (\textbf{dephasing}). As a result, pure quantum states evolve into classical statistical mixtures, and the system's ability to exhibit quantum interference is diminished. The characteristic time scale for this process, known as the \emph{coherence time}, is of utmost importance in experiments, especially in high-resolution spectroscopy, quantum information processing, and quantum optics.

\noindent
In addition to dephasing, the system-environment interaction leads to \textbf{dissipation} or \textbf{thermalization}, where energy is exchanged between the system and its surroundings, causing the system to relax toward a thermal state determined by the environment's temperature.

\noindent
Understanding and controlling these effects is crucial for the design of quantum technologies, such as quantum computers and sensors, where environmental noise can rapidly destroy fragile quantum states~\cite{laddetal2010quantumcomputers}.  \todoref{find better ref schlosshauer2007decoherencebook}

\noindent
In summary, open quantum systems theory provides the framework to describe how quantum systems lose their "quantumness" and transition toward classical behavior due to unavoidable interactions with their environment.


\subsubsection{Overview of Theoretical Approaches}
\label{subsec:overview_theoretical_approaches_oqs}

\noindent
A variety of theoretical frameworks have been developed to describe the dynamics of open quantum systems, each with its own range of validity and underlying assumptions. Some are presented here:

\noindent
In \textbf{Markovian dynamics} it is assumed that the environment has no memory: the future of the system depends only on its present state. This is valid when the environmental correlation time is much shorter than the system's characteristic timescale \todoref{find reference}. 

\todoidea{Maybe not needed:} 
\noindent
In contrast, in \textbf{non-Markovian dynamics} the memory effects of the environment are taken into account. The environment retains information about the system's past, leading to feedback and more complex evolution~\cite{breuerpetruccione2009theoryopenquantum, rivasetal2014quantumnonmarkovianitycharacterization}.
\noindent
In Time-convolutionless and Nakajima–Zwanzig master equations memory kernels describe such effects~\cite{breuerpetruccione2009theoryopenquantum, rivasetal2014quantumnonmarkovianitycharacterization}. The hierarchical equations of motion (HEOM) provide a numerically exact framework for strong coupling and non-Markovian, condensed-phase environments like liquids or solids~\cite{tanimura2020numericallyexactapproach}. This is the most computationally expensive method. Also Path-integral methods can be noted. They are based on the Feynman–Vernon influence functional and offer a non-perturbative route to non-Markovian dynamics~\cite{weiss2012quantumdissipativesystems}.


\paragraph{Stochastic Approaches}

\noindent
Stochastic Schrödinger equations and quantum trajectories unravel master equations into individual pure state evolutions ~\cite{vogtetal2013stochasticblochredfieldtheory, breuerpetruccione2009theoryopenquantum, carmichael1993opensystemsapproach}. These methods are particularly useful for simulations, as the desired precision decides the number of trajectories needed.

\paragraph{Master Equation Approaches}

\noindent
The Lindblad master equation gives the most general form for completely positive, trace-preserving dynamics under the Born-Markov approximation and is widely used in quantum optics and quantum information~\cite{breuerpetruccione2009theoryopenquantum, lindblad1976generatorsquantumdynamical}.
This thesis is focused on using a more general master equation—the Redfield equation. It will be derived in the next section \ref{subsec:Derivation_redfield_eq}.


\section{The Redfield Equation: A Central Tool}
\label{sec:Redfield_eq}

Originally developed by A.G. Redfield in 1957 and 1965 for nuclear magnetic resonance relaxation phenomena~\cite{redfield1965theoryrelaxationprocesses}, the Redfield equation describes the time evolution of the reduced density matrix (formally defined in section \ref{subsubsec:preliminaries_tools}) of a quantum system weakly coupled to a thermal environment.

\noindent
Unlike phenomenological approaches that introduce dissipation and dephasing \emph{ad hoc}, the Redfield equation relates them to environmental correlation functions or spectral densities ~\cite{breuerpetruccione2009theoryopenquantum, weiss2012quantumdissipativesystems}.

\noindent
The following requirements must be fulfilled by the final derived Redfield equation:

\begin{enumerate}
	\item The equation should be linear in the system density matrix $\dot{\rho}_S(t) = F(\rho_S(t))$ (reduced equation of motion).
	\item The equation should be Markovian, meaning that the evolution of the system density matrix  at time $t$ only depends on the state of the system at time $t$ and not on its past history.
	\item The equation should be trace-preserving, meaning that $\mathrm{Tr}[\rho_S(t)] = \mathrm{Tr}[\rho_S(0)]$ for all times $t$.
\end{enumerate}

It is important to note that the Redfield equation, while preserving trace and Hermiticity, does not guarantee complete positivity of the density matrix—a fundamental requirement for physical quantum states~\cite{rivasetal2010markovianmasterequations}. 
So care must be taken when determining when the Redfield equation is useful ~\cite{redfield1965theoryrelaxationprocesses, rivasetal2014quantumnonmarkovianitycharacterization, lietal2018conceptsquantumnonmarkovianity}.
\todoidea{explain why we choose the Redfield equation for this thesis, despite the limitations ->  it represents the best compromise between accuracy and computational efficiency for our systems of interest. -> Mostly used in biological systems and quantum chemistry}


\noindent
In this section, we will derive the Redfield equation starting from the fundamental microscopic dynamics of the system-environment composite and demonstrate how it emerges as an effective description for the reduced system dynamics. We will then examine the environmental correlation functions and spectral densities that characterize the bath properties and determine the system's relaxation behavior.

%------------------------------------------------------------------------------
%	SECTION 2: Derivation of the Redfield Equation
%------------------------------------------------------------------------------
\subsection{Derivation from Microscopic Dynamics}
\label{subsec:Derivation_redfield_eq}

\noindent
This derivation follows the approach presented in \cite{manzano2020shortintroductionlindblad} and can also be found in standard textbooks such as Breuer and Petruccione \cite{breuerpetruccione2009theoryopenquantum}. \todoidea{However here we will be more detailed and explicit in the steps.}

\subsubsection{Mathematical Preliminaries}
\label{subsubsec:preliminaries_tools}

\noindent
First we need a few mathematical tools and definitions that will be used in the derivation.


\paragraph{Density matrix formalism.}

\todoimp{please redo this text}
The density matrix formalism provides a powerful framework to describe the dynamics of quantum systems, especially when dealing with mixed states and decoherence effects.
The density matrix \(\rho\) is defined as \cite{campaiolietal2024quantummasterequations}:

\begin{equation}
	\rho = \sum_i p_i |\psi_i\rangle \langle \psi_i|, \quad p_i \geq 0, \quad \sum_i p_i = 1.
	\label{eq:DensityMatrix}
\end{equation}


\noindent
This represents a statistical mixture, the average of an ensemble with different realization of pure quantum states $\psi_i $ with classical probabilities $ p_i $. In other words one cannot tell with certainty the quantum state of the system. This is particularly useful in the context of open quantum systems, where the system is entangled with an environment and one loses information of the system of interest to the total system.
The conditions in Eq. \eqref{eq:DensityMatrix} ensure the following properties, that a physical density matrix must satisfy :
\begin{align}
	\label{eq:physical_dms}
	\rho^\dagger &= \rho, \\
	\langle \psi | \rho | \psi \rangle &\geq 0 \quad \forall |\psi\rangle, \\
	\mathrm{Tr}(\rho) &= 1. 
\end{align}

\noindent
In this formalism, for a canonical basis $\{|i\rangle\}$, representing a set of quantum states, the diagonal elements $\rho_{ii} $ represent populations and the off-diagonal elements $\rho_{ij} $ represent coherences between states.
Coherences are phase relations between different quantum states, which are crucial for interference.
For example for a two level system a clear phase relation and a pure quantum state would have off diagonal elements of $ \rho_{ij} = 1/2$.
This state is often referred to as a coherent superposition of the two states.

When coupling a system to an environment, the environment is responsible for decoherence.
The state evolves over time to a purely statistical mixture of states.

The mean value (expectation values) of an operator $O$ can be calculated by additionally averaging over the classical probabilities $p_i$ to obtain

\begin{align}
\langle A \rangle &= \sum_i p_i \langle \psi_i | A | \psi_i \rangle \\
&= \sum_n \sum_i p_i \langle \psi_i | A | \phi_n \rangle \langle \phi_n | \psi_i \rangle \\
&= \sum_n \langle \phi_n | \sum_i |\psi_i\rangle p_i \langle \psi_i | A | \phi_n \rangle \\
&= \mathrm{Tr}(\rho A).
\end{align}

\noindent
Here, an arbitrary basis $\{|\phi_n\rangle\}$ and an according completeness relation $\sum_n |\phi_n\rangle \langle \phi_n| = \mathds{1}$ was used.


\paragraph{Partial trace.}

\noindent
The partial trace is a way to simplify a quantum system's description by removing certain parts through averaging. It's like the reverse of combining spaces with a tensor product. This is handy when only one part of a combined system is of interest \cite{lambertetal2024qutip5quantum}.
For a bipartite Hilbert space $\mathcal{H_A} \otimes \mathcal{H_B}$, the partial trace over $\mathcal{H_B}$ is the unique linear map $\mathrm{Tr}_B: \mathcal{B}_1(\mathcal{H_A} \otimes \mathcal{H_B}) \to \mathcal{B}_1(\mathcal{H_A})$ satisfying $\mathrm{Tr}[(\mathrm{Tr}_B X) A] = \mathrm{Tr}[X (A \otimes \mathds{1}_B)]$ for all $A \in \mathcal{B}(\mathcal{H_A})$, where  $\mathcal{B}(\mathcal{H})$ is the algebra of bounded operators on $\mathcal{H}$.
and $\mathcal{B}_1$ denotes trace-class operators, namely those $X$ with finite trace norm 
\[ 
  \| X \|_1 = \operatorname{Tr} \sqrt{X^\dagger X} 
\]

\noindent
This ensures expectation values of local observables $A \otimes \mathds{1}_B$ are preserved. In an orthonormal basis $\{|b_i\rangle\}$ of $\mathcal{H_B}$, it acts as \cite{steebhardy2018problemssolutionsquantum}:

\begin{equation} \label{eq:ho_partial_trace_definition}
	\mathrm{Tr}_B X = \sum_i (\mathds{1}_A \otimes \langle b_i|) \, X \, (\mathds{1}_A \otimes |b_i\rangle).
\end{equation}

\noindent
With this definition thermal expectation values take the form

\begin{equation} \label{eq:ho_expectation_value} \langle A \rangle = \mathrm{Tr}[\rho A] = \frac{1}{Z} \sum_n e^{-\beta E_n} A_{nn}
\end{equation}

\begin{equation} \label{eq:ho_beta_definition}
	\beta = \frac{1}{k_{\mathrm{B}} T}
\end{equation}


\subsection{Setup: System + Environment}

\noindent
We consider a quantum system of interest interacting with an environment, which has infinite degrees of freedom. The total Hilbert space $\mathcal{H}_T$ is the tensor product of the Hilbert of the system $\mathcal{H}_S$ and the one of the environment $\mathcal{H}_E$:

\begin{equation}
	\mathcal{H}_T = \mathcal{H}_S \otimes \mathcal{H}_E.
	\label{eq:Total_Hilbert_Space}
\end{equation}

\noindent
The evolution of the total system is governed by the Liouville–von Neumann equation:

\begin{equation}
	\dot{\rho}_T(t) = -i[H_T, \rho_T(t)],
	\label{eq:Von_Neumann_Equation}
\end{equation}

\noindent
where $\rho_T(t)$ is the density matrix of the total system, $H_T$ is the total Hamiltonian, and we use units where $\hbar = 1$. This convention will be used throughout this thesis. This is the correspondence of the Schrödinger equation Eq.\eqref{eq:SchrödingerEquation} for more general classically mixed states (see section \ref{subsubsec:preliminaries_tools})

\begin{equation}
	\rho = \sum_i p_i |\psi_i\rangle \langle \psi_i|, \quad p_i \geq 0, \quad \sum_i p_i = 1.
\end{equation}

\noindent
Without loss of generality, the total Hamiltonian $H_T \in \mathcal{B}(\mathcal{H}_T)$ can be decomposed as:

\begin{equation}
	H_T = H_S \otimes \mathds{1}_E + \mathds{1}_S \otimes H_E + \alpha H_I,
	\label{eq:Total_Hamiltonian}
\end{equation}

\noindent
where $H_S$ and $H_E$ act on the individual Hilbert spaces $\mathcal{H}_S$ and $\mathcal{H}_E$, respectively, while $H_I$ acts on the composite space $\mathcal{H}_T$ (system--environment interaction). The dimensionless parameter $\alpha$ is the coupling strength parameter.

\noindent The interaction Hamiltonian is typically written in the form:

\begin{equation}
	H_I = \sum_i S_i \otimes E_i,
	\label{eq:Interaction_Hamiltonian}
\end{equation}

\noindent
where $S_i$ are system operators and $E_i$ are environment operators. These will be concretized later in Sec.~\ref{sec:harmonic_oscillator_baths}.

\subsection{Interaction Picture}

\noindent
To describe the system dynamics, we move to the interaction picture where the operators evolve with respect to $H_S + H_E$. Any arbitrary operator $O$ in the Schrödinger picture takes the form:

\begin{equation}
	\hat{O}(t) = e^{i(H_S+H_E)t} O e^{-i(H_S+H_E)t},
	\label{eq:Interaction_Picture_Operators}
\end{equation}

\noindent
in the interaction picture. The ($\hat{}$) symbol indicates that we are now working in the interaction picture. States now evolve only according to the interaction Hamiltonian $H_I$, and Eq.~\eqref{eq:Von_Neumann_Equation} becomes:

\begin{equation}
	\dot{\hat{\rho}}_T(t) = -i \alpha [\hat{H}_I(t), \hat{\rho}_T(t)],
	\label{eq:LiouvilleVN_interaction_pic}
\end{equation}

\noindent
which can be formally integrated as:

\begin{equation}
	\hat{\rho}_T(t) = \hat{\rho}_T(0) - i \alpha \int_0^t ds [\hat{H}_I(s), \hat{\rho}_T(s)].
	\label{eq:Formal_Integration}
\end{equation}

\noindent
Inserting this back into Eq.~\eqref{eq:LiouvilleVN_interaction_pic} yields:

\begin{equation}
	\dot{\hat{\rho}}_T(t) = -i \alpha \left[ \hat{H}_I(t), \hat{\rho}_T(0) \right]
	- \alpha^2 \int_0^t \left[ \hat{H}_I(t), \left[ \hat{H}_I(s), \hat{\rho}_T(s) \right] \right] ds,
	\label{eq:Second_Order_Expansion}
\end{equation}

\noindent
The iteration can be repeated, leading to a series expansion in powers of $\alpha$:

\begin{equation}
	\dot{\hat{\rho}}_T(t) = -i \alpha \left[ \hat{H}_I(t), \hat{\rho}_T(0) \right]
	- \alpha^2 \int_0^t \left[ \hat{H}_I(t), \left[ \hat{H}_I(s), \hat{\rho}_T(s) \right] \right] ds + \mathcal{O} (\alpha^3).
	\label{eq:Second_Order_Expansion_truncated}
\end{equation}

\noindent
We truncate at second order, justified by the weak coupling assumption ($\alpha \ll 1$), which represents the \textbf{Born approximation}.

\noindent
Equation~\eqref{eq:Second_Order_Expansion_truncated} still contains the full history through $\hat{\rho}_T(s)$ inside the integral and is therefore \emph{non-Markovian}. We now assume that $\hat{\rho}_T$ is approximately constant on the bath correlation time by replacing $\hat{\rho}_T(s) \to \hat{\rho}_T(t)$ inside the integrand, which represents a time-convolutionless approximation. Doing so yields

\begin{equation}
	\dot{\hat{\rho}}_T(t) = -i \alpha \big[ \hat{H}_I(t), \hat{\rho}_T(0) \big]
	- \alpha^2 \int_0^t ds\, \big[ \hat{H}_I(t), [ \hat{H}_I(s), \hat{\rho}_T(t)] \big],
	\label{eq:Second_Order_Expansion_wo_third}
\end{equation}

\noindent
which is now local in $\hat{\rho}_T(t)$ (but still retains an explicit upper integration limit $t$ so is not yet Markovian). This step will be further justified once we restrict attention to the \emph{reduced} dynamics and invoke separation of time scales.

\noindent
The exact equation of motion for $\rho(t)$ involves the full many-body dynamics of the environment, which is generally intractable. That's why the next step is to trace out the environmental degrees of freedom.


\subsection{Reduced Dynamics and Born Approximation}
\label{subsec:reduced_dynamics_born_approximation}

\noindent
Since we are interested in the dynamics of the system alone, we define the reduced density matrix using the partial trace operation defined in Eq.~\eqref{eq:ho_partial_trace_definition}:

\begin{equation}
	\rho_S(t)= \mathrm{Tr}_E[\rho_T(t)].
	\label{eq:Reduced_Density_Matrix}
\end{equation}

Thus Eq.~\eqref{eq:Second_Order_Expansion_wo_third} becomes
\begin{equation}
	\dot{\hat{\rho}}_S(t) = -i \, \alpha \, \mathrm{Tr}_E\big[\,\hat{H}_I(t),\hat{\rho}_T(0)\,\big]
	- \alpha^2 \int_0^t ds\, \mathrm{Tr}_E \big[\,\hat{H}_I(t), [\hat{H}_I(s), \hat{\rho}_T(t)] \,\big] \notag
	\label{eq:Reduced_Density_Matrix_Evolution}
\end{equation}

\paragraph{Eliminating the first-order contribution.}

\noindent
The first (order-$\alpha$) term in Eq.~\eqref{eq:Reduced_Density_Matrix_Evolution} will vanish after tracing over the environment if the interaction operators satisfy 
\begin{equation}
	\langle E_i \rangle_0 \equiv \mathrm{Tr}_E[E_i \, \hat{\rho}_E(0)] = 0,
	\label{eq:Zero_Mean_Condition}
\end{equation}

\noindent
This is achieved by assuming, that the total system starts in a \textbf{separable} product state of the form:

\begin{equation}
	\hat{\rho}_T(0) = \hat{\rho}_S(0) \otimes \hat{\rho}_E(0),
	\label{eq:Initial_Product_State}
\end{equation}

\noindent
Some textbooks impose the condition \eqref{eq:Zero_Mean_Condition} as a starting assumption, while we go a step further. When it is not initially the case one can \emph{always} enforce it by redefining the Hamiltonian through a shift:
\begin{equation}
	H_T = H_S' + H_E + \alpha H_I',
	\label{eq:Shifted_Total_Hamiltonian}
\end{equation}

\noindent
with

\begin{equation}
	H_I' = \sum_i S_i \otimes E_i', \qquad E_i' = E_i - \langle E_i \rangle_0, \qquad \langle E_i \rangle_0 \equiv \mathrm{Tr}_E[E_i \hat{\rho}_E(0)],
	\label{eq:Shifted_Interaction_Hamiltonian}
\end{equation}

\noindent
and a correspondingly shifted system Hamiltonian

\begin{equation}
	H_S' = H_S + \alpha \sum_i S_i \langle E_i \rangle_0.
	\label{eq:Shifted_System_Hamiltonian}
\end{equation}

\noindent
This ``renormalization'' merely redefines system energy levels and does not affect dissipative structure; hence, without loss of generality we assume the shift performed so that $\langle E_i \rangle_0 = 0$ and the order-$\alpha$ term can be discarded after tracing.

\noindent
Using the shifted form Eqs.~\eqref{eq:Shifted_Interaction_Hamiltonian}--\eqref{eq:Shifted_System_Hamiltonian} (so that $\langle E_i \rangle_0 = 0$), the first-order term in Eq.~\eqref{eq:Reduced_Density_Matrix_Evolution} vanishes. Explicitly, for the order-$\alpha$ contribution one has
\begin{align}
	\sum_i \mathrm{Tr}_E\big[ S_i \otimes E_i, \hat{\rho}_S(0) \otimes \hat{\rho}_E(0)\big]
	 & = \sum_i \big(S_i \hat{\rho}_S(0) - \hat{\rho}_S(0) S_i\big) \, \mathrm{Tr}_E[E_i \hat{\rho}_E(0)] = 0,
	\label{eq:Trace_Relation_first_part}
\end{align}

\noindent
which implements, that a static bath mean field can be absorbed into $H_S$.

\noindent
The reduced equation of motion becomes:

\begin{align}
	\dot{\hat{\rho}}_S(t) & = -i \, \alpha \, \mathrm{Tr}_E\big[\,\hat{H}_I(t),\hat{\rho}_T(0)\,\big]
	- \alpha^2 \int_0^t ds\, \mathrm{Tr}_E \big[\,\hat{H}_I(t), [\hat{H}_I(s), \hat{\rho}_T(t)] \,\big] \notag                     \\
	                      & = - \alpha^2 \int_0^t ds\, \mathrm{Tr}_E \big[\,\hat{H}_I(t), [\hat{H}_I(s), \hat{\rho}_T(t)] \,\big].
	\label{eq:Partial_Trace_Derivation}
\end{align}

\noindent
To make the time-translation structure explicit, set $s' = t-s$ (equivalently, $ds' = -ds$). When $s$ runs from $0$ to $t$, $s'$ runs from $t$ to $0$. Reversing the limits removes the minus sign, so the integration range remains $[0,t]$. Using this change and expanding the double commutator,
\begin{equation}
	\label{eq:double_comm_expansion_rule}
	\big[ A, [B, X] \big] = A B X - A X B - B X A + X B A,
\end{equation}

\noindent
we can rewrite Eq.~\eqref{eq:Partial_Trace_Derivation} in the form

\begin{align}
	\dot{\rho}_S(t) & = \alpha^2 \int_0^t ds \, \mathrm{Tr}_E \bigg\{
	\hat{H}_I(t) \big[ \hat{H}_I(t-s) \hat{\rho}_T(t) - \hat{\rho}_T(t) \, \hat{H}_I(t-s) \big] \notag                                   \\
	                & \qquad\qquad\qquad\; - \big[ \hat{H}_I(t-s) \hat{\rho}_T(t) - \hat{\rho}_T(t) \, \hat{H}_I(t-s) \big] \hat{H}_I(t)
	\bigg\}.
	\label{eq:Second_Order_Final_Expression}
\end{align}

\noindent
We now have to tighten the assumption of Eq. \eqref{eq:Initial_Product_State} and force that throughout the evolution the total state remains close to a factorized form $\hat{\rho}_T(t) \approx \hat{\rho}_S(t) \otimes \hat{\rho}_E(t)$. This can again be stated as the \textbf{Born approximation} of weak coupling.

\noindent
By collecting all system operators and all environment operators, and inserting the interaction Hamiltonian Eq.~\eqref{eq:Interaction_Hamiltonian} explicitly, tracking the operators at time $t - s$ with $i'$ and at time $t$ with $i$, we obtain:

\begin{align}
	\dot{\hat{\rho}}_S(t) & = \alpha^2  \sum_{i, i'} \int_0^t ds
	\bigg\{
	\mathrm{Tr}_E \big[ \hat{S}_i(t) \hat{S}_{i'}(t-s) \hat{\rho}_S(t)      \otimes   \hat{E}_{i}(t) \hat{E}_{i'}(t-s) \hat{\rho}_E(t)  \big] -  \notag                         \\
	                      & \mathrm{Tr}_E \big[ \hat{S}_i(t) \hat{\rho}_S(t) \hat{S}_{i'}(t-s)      \otimes   \hat{E}_{i}(t) \hat{\rho}_E(t) \hat{E}_{i'}(t-s)  \big] - \notag \\
	                      & \mathrm{Tr}_E \big[ \hat{S}_{i'}(t-s) \hat{\rho}_S(t) \hat{S}_i(t)      \otimes   \hat{E}_{i'}(t-s) \hat{\rho}_E(t) \hat{E}_{i}(t)  \big] +  \notag \\
	                      & \mathrm{Tr}_E \big[ \hat{\rho}_S(t) \hat{S}_{i'}(t-s) \hat{S}_i(t)      \otimes   \hat{\rho}_E(t) \hat{E}_{i'}(t-s) \hat{E}_{i}(t)  \big]
	\bigg\}.
	\label{eq:Interaction_Hamiltonian_Expansion}
\end{align}

\noindent
Since the trace only acts on the environment, the system operators can be taken out of the trace, and we define the two-point correlation functions with Eq.~\eqref{eq:ho_expectation_value}:

\begin{equation}
	C_{ii'}(t, s) \equiv \langle \hat{E}_{i}(t) \hat{E}_{i'}(t-s) \rangle = \mathrm{Tr}_E \big[\hat{E}_{i}(t) \hat{E}_{i'}(t-s) \hat{\rho}_E(t)\big],
	\label{eq:Environment_Correlation_Function}
\end{equation}

\noindent
The term 'two-point' means that we "measure" the correlation operators $\hat{E}_{i}$ and $\hat{E}_{i}$ at two different times.
If the bath operators are Hermitian, i.e. $\hat{E}_i^\dagger = \hat{E}_i$,
and since the bath density matrix is Hermitian $ \hat{\rho}_E(t) = \hat{\rho}_E^\dagger(t)$, we can use the cyclic property of the trace to show that

\begin{equation}
	C^{*}_{ii'}(t, s) = \tilde{C}_{ii'}(t, s) \equiv \langle \hat{E}_{i'}(t-s) \hat{E}_{i}(t) \rangle = \mathrm{Tr}_E \big[\hat{E}_{i'}(t-s) \hat{E}_{i}(t) \hat{\rho}_E(t)\big],
	\label{eq:Environment_Correlation_Function_Conjugate}
\end{equation}

\noindent
and thus finally obtain the desired form of the Redfield equation

\begin{align}
	\dot{\hat{\rho}}_S(t) = \alpha^2  \sum_{i, i'} \int_0^t ds
	\bigg\{
	C_{ii'}(t, s) \big[ \hat{S}_i(t),  \hat{S}_{i'}(t-s) \hat{\rho}_S(t) \big] + \text{H.c.}
	\bigg\}.
	\label{eq:Redfield_Equation_Non_Markovian}
\end{align}

\noindent
This form shows explicitly that the system evolution depends only on bath correlation functions evaluated along the free bath dynamics.


\subsection{Markov Approximation}
\label{subsec:markov_approximation}

\noindent
However the Eq. \eqref{eq:Redfield_Equation_Non_Markovian} is still not Markovian, since it carries an explicit integration over time $t$.
Now another approximation has to be made. Alongside the Born approximation, where correlations between system and environment are small, we also assume that any correlation of the environment decay on a timescale $\tau_E$ much shorter than a characteristic system evolution time $\tau_S$. The reduced density matrix $\rho_S(t)$ changes only on a much longer times $\tau_E \ll \tau_S$. Correlation functions $C_{ii'}(\tau)$ decay to negligible values for $\tau \gtrsim \tau_E$. Thus $ C(\tau) \approx 0$ for $\tau \gg 1$. Within the integral we may (i) replace the system operators $\hat{S}_{i'}(t-s)$ by its free interaction-picture evolution and (ii) extend the upper limit to infinity:

\begin{equation}
	\int_0^t ds\, C_{ii'}(t-s) f(s) \; \longrightarrow \; \int_0^{\infty} d\tau\, C_{ii'}(\tau) f(t), \qquad (t \gg \tau_E),
	\label{eq:Markov_extension_rule}
\end{equation}

\noindent
where we used the slow variation of $\rho_S(t)$ and the system operators \textit{only} on $\tau_E$. 
Applying this rule to Eq.~\eqref{eq:Partial_Trace_Derivation} yields the \emph{time-homogeneous} (Markovian) Redfield generator:

\begin{equation}
	\boxed{
		\dot{\hat{\rho}}_S(t) = - \alpha^2 \sum_{i,i'} \int_0^{\infty} d\tau \, \Big( C_{ii'}(\tau) [\hat{S}_i(t), \hat{S}_{i'}(t-\tau) \hat{\rho}_S(t)] + \text{H.c.}\Big).
	}
	\label{eq:Redfield_Markov_TimeLocal}
\end{equation}

\noindent
This step encapsulates the loss of memory: the generator now includes a very short time window $\tau_E$ where the system operators can be approximated as constant (frozen at time $\rho_S(t)$).


%------------------------------------------------------------------------------
%	SECTION 3: Environmental Correlation Functions and Spectral Properties  
%------------------------------------------------------------------------------
\section{Environmental Correlation Functions and Spectral Properties}
\label{sec:environmental_correlation_functions}

\noindent
Having derived the Redfield equation, we now focus on a crucial ingredient: the characterization of the environment through its correlation functions and spectral densities, which encode how environmental fluctuations drive relaxation and dephasing~\cite{breuerpetruccione2009theoryopenquantum, weiss2012quantumdissipativesystems}. These objects translate the abstract operator structure in Eq.~\eqref{eq:Redfield_Markov_TimeLocal} into calculable quantities and connect the microscopic bath model to experimentally accessible spectra.

\noindent
The bath correlation functions (defined by Eq.~\eqref{eq:Environment_Correlation_Function}) determine both the strength and characteristic time scales of the system--environment interaction and bridge quantum and classical noise descriptions. In what follows we show how they arise, state their properties, introduce their Fourier (spectral) representation, and explain how emission and absorption processes are simultaneously encoded.


\noindent
\todoidea{Already included? The function typically decays on the bath correlation time $\tau_E$, a prerequisite for the Markov approximation when $\tau_E$ is much shorter than the relevant system evolution time scales.}

\noindent
For additionally having a stationary bath, $[H_E, \rho_E(0)]=0$, so in the interaction picture $\hat{\rho}_E(t)=\hat{\rho}_E(0)=\rho_E(0)$ the correlators depend only on the time difference $\tau= t - s$ since one operator, e.g. $\hat{E}_{i}$ can always be written as $\hat{E}_{i}(0) = e^{-i H_E t} \hat{E}_i(t) e^{+i H_E t}$. So here we have the useful condition:
\begin{equation}
	C^{*}_{ii'}(-\tau) = C_{ii'}(\tau).
\end{equation}


\subsection{Fourier Representation and Secular approximation}
\label{subsec:Fourier_and_secular}

\noindent
For a stationary, thermal bath, where its state is given by the Gibbs state (see Sec.~\ref{subsec:bosonic_environment_gibbs}), the Kubo--Martin--Schwinger (KMS) in time domain  correlation function and its complex conjugate being related by a shift into the complex time plane:

\begin{equation}
	C_{ii'}(\tau) \equiv \langle \hat{E}_{i}(\tau) \hat{E}_{i'}(0) \rangle  =  \langle \hat{E}_{i}(0) \hat{E}_{i'}(\tau + i \beta) \rangle  \equiv C_{i'i}(+\tau + i \beta),
	\label{eq:kms_time_relation}
\end{equation}

\noindent
A short proof of this relation in the eigenbasis of the bath Hamiltonian is as follows:

\begin{align}
	C_{ii'}(\tau + i \beta) &= \mathrm{Tr}(\rho_\beta E_i E_{i'}(\tau + i \beta)) \notag \\
	&= \frac{1}{Z} \sum_{m,n} e^{-\beta E_n} \langle n | E_i | m \rangle \langle m | E_{i'}(\tau + i \beta) | n \rangle \notag \\
	&= \frac{1}{Z} \sum_{m,n} e^{-\beta E_n} e^{i(E_m - E_n)(\tau + i \beta)} (E_i)_{nm} (E_{i'})_{mn} \notag \\
	&= \frac{1}{Z} \sum_{m,n} e^{-\beta E_m} e^{i(E_n - E_m)\tau} (E_{i'})_{nm} (E_i)_{mn} \notag \\
	&= C_{i'i}(\tau).
\end{align}



\paragraph{Eigenoperator (frequency) decomposition.}

\noindent
To proceed analytically and to connect with relaxation pathways, we decompose the system coupling operators into eigenoperators of the system Hamiltonian. Let $H_S = \sum_{\epsilon} \epsilon \, \Pi_{\epsilon}$ be the spectral resolution with projectors $\Pi_{\epsilon}$. Define the Bohr frequencies $\omega = \epsilon' - \epsilon$ and the corresponding interaction operators in the eigenbasis

\begin{equation}
	S_i(\omega) = \sum_{\epsilon' - \epsilon = \omega} \Pi_{\epsilon} S_i \Pi_{\epsilon'}.
	\label{eq:Eigenoperator_Decomposition}
\end{equation}

\noindent
In the interaction picture these acquire simple oscillatory phases:

\begin{equation}
	\hat{S}_i(t) = \sum_{\omega} e^{-i \omega t} S_i(\omega), \qquad \hat{S}_i(t-\tau) = \sum_{\omega'} e^{-i \omega'(t-\tau)} S_i(\omega').
	\label{eq:Interaction_Picture_Eigenoperators}
\end{equation}

\noindent
Note, that we have now assumed a time independent system Hamiltonian $H_S$. The derivation can in principle be extended to time-dependent $H_S(t)$. \todoimp{add a reference -> mention spectroscopy (time dependent field coupling)}

\noindent
Inserting Eq.~\eqref{eq:Interaction_Picture_Eigenoperators} into Eq.~\eqref{eq:Redfield_Markov_TimeLocal} produces sums over oscillatory factors $e^{-i(\omega - \omega') t}$ multiplying integrals of $C_{ii'}(\tau) e^{i \omega' \tau}$. The inner commutator in Eq.~\eqref{eq:Redfield_Markov_TimeLocal} becomes

\begin{equation}
[\hat{S}_i(t),\, \hat{S}_{i'}(t-\tau)\rho]
= \sum_{\omega,\omega'} e^{-i(\omega - \omega')t} e^{+i \omega' \tau}
\big[ S_i(\omega),\, S_{i'}(\omega') \rho \big],
\label{eq:Inner_Commutator_Frequency_Decomposed}
\end{equation}

\noindent
and the total Redfield equation reads

\begin{equation}
	\dot{\hat{\rho}}_S(t) = -\alpha^2 \sum_{i,i'} \sum_{\omega,\omega'} e^{-i(\omega - \omega')t}
	\left( \int_0^{\infty} d\tau\, C_{ii'}(\tau)\, e^{+i \omega' \tau} \right)
	\big[ S_i(\omega),\, S_{i'}(\omega') \hat{\rho}_S(t) \big]
	+ \text{H.c.}
	\label{eq:Redfield_Frequency_Decomposed}
\end{equation}

\noindent
Now we express the correlator in frequency domain and define the noise-power spectrum of the environment \cite{lambertetal2024qutip5quantum}.

\begin{equation}
	\tilde{S}(\omega) \equiv \int_{-\infty}^{\infty} d\tau\, C(\tau)\, e^{+i \omega \tau},
	\label{eq:Noise_Power_Spectrum}
\end{equation}

\noindent
not to be confused with the system operators $S_i$.
This quantity tells how much each frequency of the bath fluctuates and contributes to noise. 
By the Wiener–Khinchin theorem (in the classical limit) it is real and positive, $\tilde{S}(\omega) \geq 0$ \todoref{find a source}\todoidea{also tells when this actually happens}.
The KMS condition Eq. \eqref{eq:kms_time_relation} can now be translated to frequency domain, where it imposes the detailed-balance symmetry

\begin{equation}
	\tilde{S}(-\omega) = e^{-\omega \beta} \tilde{S}(\omega),
	\label{eq:kms_spectral_relation}
\end{equation}

\noindent
which ensures that upward and downward transition rates obey Boltzmann ratios.


\vspace{1em}
\noindent
The exact one-sided Fourier (Laplace-) transform of the bath correlation function can be split into the power spectrum and an energy shift $ \lambda(\omega)$ as

\begin{align}
	\chi_{ii'}(\omega) &\equiv \int_0^{\infty} d\tau\, C_{ii'}(\tau)\, e^{+i \omega \tau} \\
					   &= \tfrac{1}{2}\tilde{S}_{ii'}(\omega)+i\,\lambda_{ii'}(\omega), \\
\end{align}

\noindent
with 
\begin{equation}
	\tilde{S}_{ii'}(\omega)=\chi_{ii'}(\omega)+\chi_{i'i}^*(\omega).
	\label{eq:Redfield_Rates_Definition}
\end{equation}

\noindent
which reduce Eq. \eqref{eq:Redfield_Frequency_Decomposed} to the compact form

\begin{equation}
	\boxed{
	\dot{\hat{\rho}}_S(t)
	= -\alpha^2 \sum_{i,i'} \sum_{\omega,\omega'} e^{-i(\omega - \omega')t}
	\, \chi_{ii'}(\omega') \,
	\big[ S_i(\omega),\, S_{i'}(\omega') \hat{\rho}_S(t) \big]
	+ \text{H.c.}
	}
\end{equation}

\noindent
Going back to the Schrödinger picture, this yields

\begin{equation}
\dot{\rho}_S(t)=-i[H_S+H_{\rm LS},\rho_S(t)]+\mathcal{D}_\mathrm{R}[\rho_S(t)],
\end{equation}

\noindent
with the Lamb-shift Hamiltonian

\begin{equation}
	H_{\rm LS}=\sum_{\omega}\sum_{i, i'}\lambda_{ii'}(\omega)\,
	S_i^\dagger(\omega)S_{i'}(\omega),
	\label{eq:Lamb_Shift_Hamiltonian}
\end{equation}

\noindent
which will be disregarded in the subsequent derivation, and the Redfield dissipator

\begin{equation}
	\mathcal{D}_\mathrm{R}[\rho]
	=\frac{1}{2}\sum_{\omega,\omega'}\sum_{i, i'}
	e^{-i(\omega-\omega')t} \tilde{S}_{ii'}(\omega')\,
	\Big(
		S_{i'}(\omega')\rho S_i^\dagger(\omega)
		- S_i^\dagger(\omega)S_{i'}(\omega')\rho
	\Big) + \text{H.c.}
	\label{eq:Redfield_Dissipator}
\end{equation}


\paragraph{Matrix elements and Redfield tensor.}

\noindent
In the eigenbasis of the Hamiltonian $\{|a\rangle\}$, taking $\langle a|\,\cdot\,|b\rangle$ of the master equation and grouping terms gives

\begin{equation}
\frac{d}{dt}\rho_{ab}(t)
= -i\,\omega_{ab}\rho_{ab}(t) + \sum_{c,d} R_{abcd}\,\rho_{cd}(t),
\label{eq:bloch_redfield_basis}
\end{equation}

\noindent
where we defined the Bloch–Redfield tensor that is implemented also in QuTiP as

\begin{align}
R_{abcd}
= -\frac{1}{2} \sum_{i, i'}\Big\{&
\delta_{bd}\sum_n S^{i}_{an}S^{i'}_{nc}\,\tilde{S}_{ii'}(\omega_{cn})
- S^{i'}_{ac}S^{i}_{db}\,\tilde{S}_{ii'}(\omega_{ca})
\notag\\
&+\delta_{ac}\sum_n S^{i}_{dn}S^{i'}_{nb}\,\tilde{S}_{ii'}(\omega_{dn})
- S^{i'}_{ac}S^{i}_{db}\,\tilde{S}_{ii'}(\omega_{db})
\Big\}.
\label{eq:Rabcd_full}
\end{align}


\paragraph{Secular approximation.}

\noindent
The oscillatory prefactors $e^{-i(\omega - \omega') t}$ in Eq.~\eqref{eq:Redfield_Dissipator} average to zero on coarse-grained times $\Delta t$ satisfying $\tau_E \ll \Delta t \ll 1/|\omega - \omega'|$ whenever $\omega \neq \omega'$. 
Keeping only terms with $\omega_{ab}=\omega_{cd}$ in Eq.~\eqref{eq:Rabcd_full} removes such fast-rotating couplings and ensures a block-diagonal structure in the Redfield tensor $R_{abcd}$.

\noindent
This yields the \emph{secular} Redfield (Lindblad form) equation:

\begin{align}
	\dot{\rho}_S(t) & = -i [H_S + H_{\text{LS}}, \rho_S(t)]                                                                                                                                     \\
	                & + \sum_{i,i'} \sum_{\omega} \gamma_{ii'}(\omega) \Big( S_j(\omega) \rho_S(t) S_i^{\dagger}(\omega) - \tfrac{1}{2} \{ S_i^{\dagger}(\omega) S_j(\omega), \rho_S(t) \} \Big),
	\label{eq:Secular_Lindblad_Form}
\end{align}

\noindent
where $\gamma_{ii'}(\omega) = 2 \mathrm{Re}\,\chi_{ii'}(\omega)$. This form is guaranteed to preserve complete positivity provided the matrix $[\gamma_{ii'}(\omega)]_{i,i'}$ is positive semidefinite for each frequency $\omega$. Without the secular approximation, Eq.~\eqref{eq:Redfield_Dissipator} need not generate a completely positive dynamical map, explaining the caution required when applying the full Redfield equation.

\noindent
The corresponding tensor structure can be described as 

\begin{equation}
R_{abcd}^{(\sec)} = \Theta_\epsilon(\omega_{ab} - \omega_{cd}) R_{abcd}, \quad \Theta_\epsilon(\Delta) = \begin{cases} 1, & |\Delta| < \epsilon, \\ 0, & \text{otherwise}. \end{cases}
\end{equation}


\paragraph{Common simplification (uncorrelated Hermitian couplings).}

\noindent
If the baths are uncorrelated and $S_i=S_i^\dagger$:

\begin{align}
\tilde{S}_{ii'}(\omega) &= \delta_{ii'} S_i(\omega) \implies \notag \\
R_{abcd} &= -\frac{1}{2} \sum_{i} \Big\{
\delta_{bd} \sum_n S^{i}_{an} S^{i}_{nc} S_{i}(\omega_{cn})
- S^{i}_{ac} S^{i}_{db} S_{i}(\omega_{ca}) \notag \\
&\quad + \delta_{ac} \sum_n S^{i}_{dn} S^{i}_{nb} S_{i}(\omega_{dn})
- S^{i}_{ac} S^{i}_{db} S_{i}(\omega_{db})
\Big\}.
\label{eq:Rabcd_uncor}
\end{align}


\subsection{Physical Interpretation: Emission and Absorption Processes}
\label{subsec:physical_emission_absorption}

\noindent
Emission and absorption are both automatically included because $\tilde{S}_{ii'}(\omega)$ contains positive- and negative-frequency components related by Eq.~\eqref{eq:kms_spectral_relation}. Positive frequencies ($\omega>0$) describe system energy loss (emission), while negative frequencies ($\omega<0$) describe system energy gain (absorption). Detailed balance follows immediately: the ratio of absorption to emission contributions at frequency $\omega>0$ is $e^{- \omega \beta}$. For a single bosonic bath with scalar spectral density $J(\omega)$ (defined for $\omega>0$) one often writes the symmetrized spectrum

\begin{equation}
	S(\omega) = 2\pi J(|\omega|) \begin{cases} n_{\text{th}}(\omega)+1, & \omega>0, \\ n_{\text{th}}(|\omega|), & \omega<0, \end{cases}
	\label{eq:bose_symmetric_spectrum}
\end{equation}

\noindent
with Bose--Einstein occupation $n_{\text{th}}(\omega) = \big(e^{\omega \beta}-1\big)^{-1}$, explicitly exhibiting stimulated plus spontaneous emission $(n_{\text{th}}+1)$ versus absorption $(n_{\text{th}})$~\cite{weiss2012quantumdissipativesystems}. This formulation makes clear how both processes and their thermal weighting emerge from a single function. How the now defined spectral density function comes into play is explained for the concrete example of a bath of harmonic oscillators (this can be thought of as the electromagnetic field, solid state vibrations (phonons) or other bosonic modes) in the following section. \todoref{find sources}


%------------------------------------------------------------------------------
% SECTION 4: Harmonic Oscillator Bath and Correlation Functions
%------------------------------------------------------------------------------
\section{Harmonic Oscillator Baths and Explicit Correlation Functions}
\label{sec:harmonic_oscillator_baths}

\noindent
In the previous chapter, we derived the Redfield equation and introduced the environmental correlation functions $C_{ij}(\tau)$ that characterize the system-bath interaction. These correlation functions are central to the Redfield formalism, as they encode all the relevant information about how the environment affects the quantum system dynamics.                                                                                                         
To make the Redfield equation practically useful, we need to specify the form of these correlation functions for realistic environmental models. In this chapter, we focus on one of the most important and widely applicable environmental models: the harmonic oscillator bath. This model captures the essential physics of many real environments, from electromagnetic field fluctuations to phonon baths in solid-state systems.       


\noindent
The general task of this chapter is to calculate the bath correlation functions. This correlator measures how the operator \( E \) at two different times is correlated in the thermal bath state.
For the harmonic oscillator bath model, these correlation functions can be calculated exactly. The results will show how the bath correlation functions connect to physically observable quantities such as spectral densities and how they determine the system's approach to thermal equilibrium.

\noindent
To also do these explicit calculations, we will need the following mathematical result.


\paragraph{Geometric series.}
The infinite geometric series

\begin{equation} \label{eq:ho_infinite_geometric_series}
	S = a + ar + ar^2 + ar^3 + \dots = \sum_{n=0}^{\infty} ar^n,
\end{equation}

\noindent
converges to

\begin{equation} \label{eq:ho_geometric_series_sum}
	S = \frac{a}{1-r} \quad \text{for } |r|<1.
\end{equation}

\noindent
Differentiating Eq.~\eqref{eq:ho_geometric_series_sum} with respect to $r$ gives

\begin{equation} \label{eq:ho_derivation_geometric_sum}
	\sum_{n=0}^{\infty} n r^n = \frac{r}{(1-r)^2}, \quad |r|<1.
\end{equation}

\noindent
We now specialize the formal correlation functions introduced above to the paradigmatic and widely applicable case of a bosonic (harmonic oscillator) environment. The full scope of thermal correlation functions, spectral density representations, and standard phenomenological models (Ohmic, sub-/super-Ohmic, Drude--Lorentz) is explained.


\subsection{Bosonic Environment and Gibbs State}
\label{subsec:bosonic_environment_gibbs}

\noindent
A bosonic bath is modeled as a (large) collection of independent harmonic oscillators in thermal equilibrium:

The thermal state of the bosonic bath is given by the Gibbs density matrix:

\begin{equation} \label{eq:ho_gibbs_state}
    \rho_{Gibbs} = \frac{e^{-\beta H}}{\mathrm{Tr}[e^{-\beta H}]},
\end{equation}

\noindent
where $\beta = 1/(k_B T)$ again is the inverse temperature of the environment. The bath Hamiltonian consists of independent harmonic oscillators:

\begin{equation}
    H = \sum_k \omega_k \Big(b_k^{\dagger} b_k + \tfrac{1}{2}\Big),
	\label{eq:ho_bath_hamiltonian}
\end{equation}

\noindent
and the average energy of mode $k$ is

\begin{equation}
    E_k = \omega_k (n_k + \tfrac{1}{2}),
\end{equation}

\noindent
with $n_k = \langle b_k^{\dagger} b_k \rangle$ the Bose--Einstein occupation number.


\noindent
The zero-point contribution $E_{\text{vac}} = \tfrac{1}{2} \sum_k \omega_k$ adds only a mode-independent constant to the bath energy. Constant energy offsets commute with all observables, so they neither modify thermal populations nor enter the correlation functions that drive the system dynamics. We therefore drop this term (standard for normal ordering, where raising operators are placed to the left of lowering operators). In other words only energy \emph{differences} influence the evolution and we neglect this term from now on.



\subsubsection{Single Mode Partition Function and Occupation}
\label{subsubsec:single_mode}

\noindent
For a single mode $k$, $H=\omega_k b^{\dagger}_k b_k$, the thermal state reads

\begin{equation} \label{eq:ho_single_mode_density_matrix}
	\rho = \frac{e^{-\beta \omega_k b^{\dagger}_k b_k}}{Z},
\end{equation}

\noindent
with partition function

\begin{align} \label{eq:ho_partition_function}
        Z_k & \equiv \mathrm{Tr}\left[e^{-\beta H}\right] = \sum_{m=0}^{\infty} \langle m | e^{-\beta \hbar \omega_k (n_k + \frac{1}{2})} | m \rangle \\
        & = \frac{e^{-\beta \hbar \omega_k / 2}}{1 - e^{-\beta \hbar \omega_k}}.
\end{align}

\noindent
The Bose--Einstein average occupation number follows using Eq.~\eqref{eq:ho_derivation_geometric_sum}:

\begin{align} \label{eq:ho_expectation_number_operator}
        n_k & = \langle b_k^{\dagger} b_k \rangle_{\text{th}} = \mathrm{Tr} \left[ b_k^{\dagger} b_k \frac{e^{-\beta H}}{Z_k} \right]\\
                                           & = \frac{\mathrm{Tr} \left[ b_k^{\dagger} b_k e^{-\beta \hbar \omega_k b_k^{\dagger} b_k} \right]}{Z_k}\\
                                           & = \frac{\sum_{m=0}^{\infty} \langle m|b_k^{\dagger} b_k e^{-\beta \hbar \omega_k b_k^{\dagger} b_k}|m \rangle}{\frac{e^{-\beta \hbar \omega_k / 2}}{1 - e^{-\beta \hbar \omega_k}}} \\%    &= \frac{e^{-\beta \hbar \omega_k}}{1 - e^{-\beta \hbar \omega_k}} \\
            & = \frac{e^{-\beta \hbar \omega_k}}{1 - e^{-\beta \hbar \omega_k}}\\
            & = \frac{1}{e^{\beta \hbar \omega_k} - 1}.
\end{align}
\vspace{1em}
\noindent
Since the bath is composed of many independent modes, this calculation can be done for every mode, and the total partition function factorizes to

\begin{equation} \label{eq:ho_generalized_partition_function}
	Z_{\text{bath}} = \prod_k Z_k = \prod_k \frac{e^{-\beta \omega_k /2}}{1 - e^{-\beta \omega_k}}.
\end{equation}

\noindent
Since now the thermal properties of the system alone are known, we can focus on the coupling it to a small system of interest. Therefore we explicitly calculate the bath correlation functions from Eq.~\eqref{eq:Environment_Correlation_Function} for the simple but arguably the most important case of a single coupling operator $E$. The derivation is followed and expanded a bit from \cite{thingnasteadystatetransportproperties}.



\subsection{Microscopic Form of the Bath Correlator}
\label{subsec:microscopic_bath_correlator}

\noindent
With linear system--bath coupling to oscillator displacements (Caldeira--Leggett model \cite{hagstrommorrison2011caldeiraleggettmodel}) we write the single bath coupling operator as

\begin{equation} \label{eq:ho_bath_operator}
	E = \sum_{n=1}^{\infty} c_n x_n, \qquad x_n = \sqrt{\frac{1}{2 m_n \omega_n}} (b_n + b_n^{\dagger}).
\end{equation}

\noindent
In the interaction picture the bath operators evolve as $E(t) = e^{i H_E t} E e^{-i H_E t}$ with $H_E$ from Eq.~\eqref{eq:ho_bath_hamiltonian}, yielding

\begin{align}
	\hat{E}(0)    & = \sum_{n} c_n \sqrt{\frac{1}{2 m_n \omega_n}} (b_n + b_n^{\dagger}), 
	\label{eq:ho_bath_operator_t0}                                                   \\
	\hat{E}(\tau) & = \sum_{n} c_n \sqrt{\frac{1}{2 m_n \omega_n}} \Big(b_n e^{-i \omega_n \tau} + b_n^{\dagger} e^{i \omega_n \tau}\Big). \label{eq:ho_bath_operator_tau}
\end{align}

\noindent
The thermal correlation function  for this single relevant bath operator is

\begin{equation} \label{eq:ho_bath_correlator}
	C(\tau) = \langle \hat{E}(\tau) \hat{E}(0) \rangle.
\end{equation}

\noindent
Inserting Eqs.~\eqref{eq:ho_bath_operator_t0},\eqref{eq:ho_bath_operator_tau} yields a double sum over bath modes,

\begin{align}
	C(\tau) &= \sum_{n,m} \frac{c_n c_m}{\sqrt{4 m_n m_m \omega_n \omega_m}} \Big\langle \big(b_n e^{-i \omega_n \tau} + b_n^{\dagger} e^{i \omega_n \tau}\big) (b_m + b_m^{\dagger}) \Big\rangle, \\[0.3em]
	&= \sum_{n,m} \frac{c_n c_m}{\sqrt{4 m_n m_m \omega_n \omega_m}} \Big( e^{-i \omega_n \tau} \langle b_n b_m^{\dagger} \rangle + e^{i \omega_n \tau} \langle b_n^{\dagger} b_m \rangle \Big),
	\label{eq:ho_correlator_intermediate}
\end{align}

\noindent
where mixed terms such as $\langle b_n b_m \rangle$ vanish for a thermal state of uncoupled oscillators.
Doing similar calculations to Eqs.~\eqref{eq:ho_expectation_number_operator} gives
\begin{equation}
	\langle b_n b_m^{\dagger} \rangle = \delta_{nm}(n_n+1) \quad \text{and} \quad \langle b_n^{\dagger} b_m \rangle = \delta_{nm} n_n,
\end{equation}

\noindent
with $n_n$ given by Eq.~\eqref{eq:ho_expectation_number_operator}. Using these results we obtain the discrete form of the bath correlation function

\begin{equation} \label{eq:ho_correlator_result}
	C(\tau) = \sum_{n} \frac{c_n^2}{2 m_n \omega_n} \Big[(n_n+1) e^{-i \omega_n \tau} + n_n e^{i \omega_n \tau}\Big].
\end{equation}


\subsection{Spectral Density Representation}
\label{subsec:ho_spectral_density}

\noindent
To connect this discrete expression with the continuum form, note that the correlator can be expanded by writing $e^{\pm i \omega_n \tau} = \cos(\omega_n \tau) \pm i \sin(\omega_n \tau)$. This yields

\[
	C(\tau) = \sum_{n} \frac{c_n^2}{2 m_n \omega_n} \Big[ (n_n+1)(\cos - i \sin) + n_n(\cos + i \sin) \Big],
\]

\noindent
where the trigonometric arguments $\omega_n \tau$ are implied. Collecting cosine and sine contributions gives

\[
	C(\tau) = \sum_{n} \frac{c_n^2}{2 m_n \omega_n} \Big[(2 n_n + 1) \cos(\omega_n \tau) - i \sin(\omega_n \tau)\Big].
\]

\noindent
Since $2 n_n + 1 = 2/(e^{\beta \omega_n}-1) + 1 = \coth(\beta \omega_n / 2)$ (recall $\coth(x/2) = (e^{x}+1)/(e^{x}-1)$), we can rewrite this as

\begin{equation}
	C(\tau) = \sum_{n} \frac{c_n^2}{2 m_n \omega_n} \Big[ \coth\Big(\frac{\beta \omega_n}{2}\Big) \cos(\omega_n \tau) - i \sin(\omega_n \tau) \Big].
\end{equation}

\noindent
Introducing the discrete spectral density


\begin{equation} \label{eq:ho_bath_spectral_density}
	J(\omega) = \pi \sum_{n} \frac{c_n^2}{2 m_n \omega_n} \delta(\omega - \omega_n),
\end{equation}

\noindent
which summarizes how strong which frequencies of the environment couple to the system. Here it is assumed that the bath has discrete modes $\omega_n$ and each couples to the system with strength $\propto c_n$. One finds that for any smooth $f(\omega)$,

\begin{equation}
	\sum_{n} \frac{c_n^2}{2 m_n \omega_n} f(\omega_n) = \frac{1}{\pi} \int_{0}^{\infty} d\omega\, J(\omega) f(\omega).
\end{equation}

\noindent
Applying this identity with $f(\omega) = \coth(\beta \omega / 2) \cos(\omega \tau) - i \sin(\omega \tau)$ yields a representation of the bath correlator in a continuum limit

\begin{equation} \label{eq:ho_correlator_spectral_density}
	C(\tau) = \int_{0}^{\infty} d\omega \, \frac{J(\omega)}{\pi} \Big[ \coth\Big(\frac{\beta \omega}{2}\Big) \cos(\omega \tau) - i \sin(\omega \tau) \Big].
\end{equation}


\noindent
The physical interpretation of this is the following.
In the macroscopic limit the mode set becomes dense, introducing a density of states $\rho(\omega)$ and form factor $g(\omega)$ yields $J(\omega)=\rho(\omega) g(\omega)^2$. Put differently, now one only has to specify a parametrization of the continuous spectral density function. In Eq.~\eqref{eq:ho_correlator_spectral_density} purely dissipative (imaginary) and noise (real, symmetrized) parts are separated.
\todoref{<- find a fitting source}

\vspace{1em}
\noindent
We have found now different representations of the bath correlation function $C(\tau)$, which is the key quantity determining the system dynamics in the Redfield equation. The spectral density $J(\omega)$ encodes how strongly each bath mode at frequency $\omega$ couples to the system, while the temperature-dependent $\coth(\beta \omega / 2)$ factor weights these contributions according to thermal occupation. The spectral density can directly be related to the noise power spectrum defined in Eq.~\eqref{eq:Noise_Power_Spectrum} as

\begin{equation} \label{eq:ho_correlator_final}
	\tilde{S}(\omega) = \int_{-\infty}^{\infty} d\tau \, C(\tau) e^{+i \omega \tau} = 2\pi J(|\omega|) \begin{cases} n_{\text{th}}(\omega)+1, & \omega>0, \\ n_{\text{th}}(|\omega|), & \omega<0, \end{cases}
\end{equation}

\noindent
So either the Correlator, or the noise power spectrum, or the spectral density, together with the Temperature $\beta$ (hidden inside $n_{\text{th}}$), can be used to fully characterize the environmental coupling to the system. The spectral density is often the most convenient choice since it is a real-valued function of a single variable $\omega$ and can be directly related to microscopic models or phenomenological forms.

\noindent
Then to characterize different environments, one only has to choose one parametrization of the spectral density $J(\omega)$ which directly defines the correlators. Now we will introduce some of the most common choices for $J(\omega)$.

\section{Common choices of Spectral Densities}
\label{sec:common_choices_spectral_densities}

\subsection{Ohmic and Related Spectral Densities}
\label{subsec:ohmic_spectral_density}

\noindent
Source: Ulrich Weiss chapter 7.3:
We have seen, that the Redfield equation is influenced by classical phenomenological models of dissipation. For an Ohmic spectral density, the damping is assumed to be frequency-independent, and the spectral density is given by:

\begin{equation} \label{eq:ohmic_spectral_density}
        J(\omega) \propto \gamma \omega,
\end{equation}

\noindent
where \( \gamma \) is the damping constant. This corresponds to a classical Friction-like dissipation. To ensure physical behavior, a cutoff is introduced:

\begin{equation} \label{eq:cutoff_spectral_density}
        J(\omega) = \eta \frac{\omega^s}{\omega_c^{s-1}} e^{-\omega / \omega_c},
\end{equation}

\noindent
which enforces convergence of integrals such as Eq.~\eqref{eq:ho_correlator_final}. Here \( \eta \) is a dimensionless coupling constant, \( \omega_c \) is the cutoff frequency, and \( s \) determines the type of spectral density (Ohmic for \( s = 1 \), sub-Ohmic for \( s < 1 \), and super-Ohmic for \( s > 1 \)) \cite{weiss2012quantumdissipativesystems, lambertetal2024qutip5quantum}.

\noindent
Another widely used model with linear low-frequency behavior is the Drude--Lorentz spectral density, which introduces a Lorentzian cutoff:

\begin{equation}
	\label{eq:dl_spectral_density}
	J(\omega) = 2 \lambda \frac{\omega \omega_c}{\omega^2 + \omega_c^2},
\end{equation}

\noindent
where $\lambda$ is the reorganization energy, and $\omega_c$ is again the cutoff frequency. This model represents an "overdamped Brownian oscillator” and is particularly relevant for systems coupled to a structured environment, such as molecular systems in a solvent or solid-state systems with phonon interactions .%\cite{}
\todoref{fact check}
\noindent
Here is a visual comparison between the ohmic and Drude--Lorentz spectral densities, along with their corresponding correlation functions and power spectra. The Drude--Lorentz model has a longer non-zero tail for large frequencies compared to the Ohmic case.

\todofix{uncomment these figures}
\iffalse
	\begin{figure}[t]
		\centering
		\includegraphics[width=\textwidth]{bath_comparison_combined_0.010_100.00_100.000.png}
		\caption{Comparison of representative bath models (Ohmic and Drude--Lorentz) showing spectral densities, associated power spectra, and time-domain correlation functions for coupling strength $\alpha = 0.1$, cutoff $\omega_c = 100$, and temperature $T=100$. Distinct spectral shapes map directly onto different relaxation and dephasing behaviors.}
		\label{fig:bath_comparison}
	\end{figure}
	% also include a temperature dep. picture
	\begin{figure}
		\centering
		\includegraphics[width=\textwidth]{temperature_analysis_ohmic_bath.png}
		\caption{Temperature dependence of the bath correlation function for an Ohmic bath with $\alpha = 1$ and cutoff $\omega_c = 100$. Higher temperatures increase the amplitude and decrease the correlation time, reflecting enhanced thermal fluctuations.}
		\label{fig:bath_temperature_comparison}
	\end{figure}
\fi

\noindent
As we can see both choices translate to one single exponentially decaying memory kernel.

\todoidea{maybe also show for underdamped Brownian oscillator }
\todoidea{mention that for more complicated systems, one could fit a sum of such spectral densities to experimental data -> qutip HEOM solvers use such generated bath objects}
% !TEX root = ../main.tex
\chapter{Principles of Spectroscopy}
\label{chapter_spectroscopy}
%-------------------------------------------------------------------------------
%	SECTION 1: INTRODUCTION TO SPECTROSCOPY
%-------------------------------------------------------------------------------

\section{Fundamentals of Spectroscopy}
\label{sec:spectroscopy_fundamentals}

\todoidea{This chapter	 should serve as a foundation by emphasizing electronic transitions, nonlinear optics, coherence phenomena, and their relevance to complex (e.g., biological) systems.
Vibrational and other spectroscopies are less relevant for this work, so they should be mentioned but not in detail.}
\todoidea{
Brief motivation: Why spectroscopy is a powerful probe of quantum dynamics.

Distinction between linear and nonlinear optical spectroscopy.

Position of two-dimensional photon-echo spectroscopy (2D PES) within this landscape.

What this chapter provides: the physical and mathematical tools needed for later modeling.
}


\noindent 
Spectroscopy, in its broadest definition, is the study of the interaction between matter and electromagnetic radiation as a function of wavelength or frequency \cite{berman2011principleslaserspectroscopy, mukamel1995principlesnonlinearoptical}.
It provides insights into the composition, structure, and dynamics of physical systems by examining how they absorb, emit, or scatter light. The fundamental principle underlying all spectroscopic methods is that each atom, molecule, or complex system has a unique set of energy levels, and transitions between these levels involve the absorption or emission of photons with specific energies \todoref{find better source}\cite{boyd2008chapter1nonlinear}. Thats why spectroscopy is a powerful probe of quantum dynamics.



\subsection{Basic Principles}
\label{subsec:basic_principles}

\noindent The foundation of spectroscopy rests on the quantization of energy in atomic and molecular systems. According to quantum mechanics, atoms and molecules can exist only in discrete energy states \cite{albashetal2012quantumadiabaticmarkovian}. The energy difference between two states, $\Delta E$, determines the frequency $\nu$ or wavelength $\lambda$ of light that can be absorbed or emitted during a transition between these states, following Planck's relation:

\begin{equation}
	\Delta E = h\nu = \frac{hc}{\lambda}
	\label{eq:planck_relation}
\end{equation}

\noindent 
where $h$ is Planck's constant and $c$ is the speed of light. The energy structure of matter can thus be probed by observing the spectrum of absorbed or emitted radiation.


%-------------------------------------------------------------------------------
%   NEW SUBSECTION: CHARACTERIZATION OF ELECTROMAGNETIC RADIATION
%-------------------------------------------------------------------------------
\subsection{Characterization of Electromagnetic Radiation}
\label{subsec:em_radiation_characterization}

\noindent 
A monochromatic electromagnetic plane-wave in free space (vacuum) can be written as

\begin{equation}
	E(\vec{r},t) = E_0 \cos(\vec{k} \cdot \vec{r} - \omega t + \phi)
	\label{eq:plane_wave}
\end{equation}

\noindent 
where $E_0$ is the (real) amplitude, $\phi$ a phase-kick, $\vec{k}$ the wavevector (spatial angular frequency), and $\omega$ the angular frequency of the wave. The real field can be decomposed into complex positive and negative frequency parts as

\iffalse
\begin{equation}
	E^{(+)} = E_0 e^{i(\vec{k} \cdot \vec{r} - \omega t + \phi)}
	\label{eq:positive_frequency_e_field}
\end{equation}

\noindent 
with $E^{(-)} = E^{(+)}^{*}$ and thus $E = 1/2 \{E^{(+)} + E^{(-)}\}$.
$E^{(+)}$ is called the positive-frequency part because in Fourier space it contains only components with frequencies $\omega > 0$ \todoref{fact check}. This will come in handy applying a rotating wave approximation as the positive field part rotates in the same direction as the rotating frame and the negative rotates in the opposite direction \cite{hamm2005principlesnonlinearoptical}.
\fi
\noindent 
The fundamental kinematic relations for propagation in vacuum are

\begin{equation}
	\omega = 2\pi\nu, \qquad |\vec{k}| = \frac{2\pi}{\lambda}, \qquad \lambda = \frac{c}{\nu} = \frac{2\pi c}{\omega}
	\label{eq:wavelength_frequency_relation}
\end{equation}

\noindent 
with $\nu$ the (cycle) frequency in hertz (Hz), $\lambda$ the wavelength. 
%In a material medium the phase velocity is $v_p = c/n(\nu)$ and $\lambda$ is replaced by $\lambda_n = v_p/\nu = \lambda/n(\nu)$, where $n(\nu)$ is the refractive index.

\noindent 
Note that the \emph{magnitude of the wavevector } $k = 2\pi/\lambda$ (units m$^{-1}$), shall not be confused with the \emph{(spectroscopic) wavenumber} $\tilde{\nu}$, defined as

\begin{equation}
	\tilde{\nu} = \frac{1}{\lambda} \quad [\mathrm{cm}^{-1}].
	\label{eq:wavenumber_definition}
\end{equation}

This quantity is commonly used in spectroscopy to avoid large powers of ten. 
Now to give a broad idea of where transitions that one would like to probe with spectroscopy lie, we can look at the typical energy scales of different types of transitions.

\subsection{Classification of Spectroscopic Techniques}
\

\todoidea{can I also include some figure illustrating this?}
\noindent
Spectroscopy can be categorized by (i) the nature of the transition (rotational, vibrational, electronic, nuclear), (ii) the number of photons involved (linear vs.\ nonlinear), and (iii) the detection scheme (absorption, emission, or scattering). 

\noindent
(i):

\noindent
Different types of transitions are probed at distinct regions of the electromagnetic spectrum, reflecting the corresponding energy and time scales and providing fundamentally different information about the system of interest. 

\noindent
Ultrafast laser development has enabled femtosecond and even attosecond spectroscopy, allowing real-time tracking of electron motion and energy flow in complex systems.

\noindent
For example, rotational transitions lie in the microwave region. Here molecular geometry is revealed through quantized rotational energy levels. In vibrational spectroscopy, bonding patterns are measured within the infrared ($\lambda\!\sim\!2.5$–$25\,\mu\mathrm{m}$, $\tilde{\nu}\!\sim\!400$–$4000\,\mathrm{cm^{-1}}$) region of the spectrum. In this thesis electronic transitions between molecular orbitals will be probed with visible frequencies of  ($\lambda\!\sim\!400$–$750\,\mathrm{nm}$, $\tilde{\nu}\!\sim\!13{,}000$–$25{,}000\,\mathrm{cm^{-1}}$), enabling studies of excited-state structure and dynamics.

\noindent
This energetic hierarchy underlies the classification of spectroscopic techniques: higher photon energies probe faster and more localized motions.
While timescales of Micro-milliseconds relate to  Chemical reactions and protein folding, Vibrational relaxation and rotational dynamics happen at picoseconds. Important for this thesis, electronic transitions and vibrational coherences occur at femtoseconds. Recent advances in attosecond spectroscopy now even allow probing real time electron dynamics within atoms and molecules \cite{rupprechtetal2025tracinglonglivedatomic}.

\subsection{Energy Dissipation After Absorption}

\noindent
After an interaction with a laser pulse, absorption of a photon, the excited molecule relax through two principal pathways:
\textbf{Non-radiative relaxation:} Internal conversion and vibrational energy transfer to the environment convert excitation energy into heat. What will be more important for this thesis is \textbf{Radiative relaxation} where the ground state is populated again. The emitted photons that are typically isotropic and thus contribute little to the forward beam.

\noindent
Absorption spectra thus record the net loss of intensity from the incident mode, while emission spectra provide complementary information on excited-state structure and relaxation dynamics. Both will be central in later sections involving heterodyne detection of emitted fields. \todoref{find ref}



\subsection{Macroscopic Samples and Collective Dipolar Oscillations}
\label{subsec:macroscopic_samples}

\noindent 
When considering spectroscopic measurements on macroscopic samples, it is essential to understand how individual molecular responses combine to produce observable signals. In a typical spectroscopic experiment, the sample contains on the order of Avogadro's number ($\sim 10^{23}$) of molecules, each potentially acting as an oscillating electric dipole when interacting with electromagnetic radiation \cite{feynman1965feynmanlecturesphysics}.

\noindent 
The collective behavior of these molecular dipoles determines the induced macroscopic polarization of the sample. When an external electromagnetic field is applied, individual molecules undergo dipole transitions, creating time-varying dipole moments $\vec{\mu}_i(t)$. The coherent superposition of these individual dipolar oscillations produces a macroscopic polarization wave $\vec{P}(\vec{r}, t)$ that can be detected experimentally

\begin{equation}
	\vec{P}(\vec{r}, t) = \frac{1}{V} \sum_{i} \vec{\mu}_i(t) \delta(\vec{r} - \vec{r}_i).
	\label{eq:macroscopic_polarization}
\end{equation}

\noindent 
where $V$ is the sample volume, $\vec{r}_i$ is the position of the $i$-th molecule, and the sum extends over all molecules in the interaction region. This polarization acts as a source term in Maxwell's equations, generating the electromagnetic fields that constitute the spectroscopic signal \cite{abramaviciusetal2009coherentmultidimensionaloptical}

\begin{equation}
	\nabla \times \nabla \times \vec{E}(\vec{r}, t) + \frac{1}{c^2} \frac{\partial^2}{\partial t^2} \vec{E}(\vec{r}, t) = - \frac{4 \pi}{c^2} \frac{\partial^2}{\partial t^2} \vec{P}(\vec{r}, t).
\end{equation}

\noindent
\todoidea{why does it hold that $E_S \propto i P_S$? -> its an approximation in the slow-varying amplitude approximation\cite{mukamel1995principlesnonlinearoptical}, cite boyd chapter 6, page 295 because of complex polarization!}

\noindent 
The phase relationships between individual dipolar oscillations are crucial for understanding spectroscopic line shapes and signal intensities \todofix{why?} . 

\todoidea{add the following to a different section / into the results}
In the case of inhomogeneously broadened systems\footnote{\todoidea{ALSO ADD: When the harmonic potential surfaces of the ground and the excited states are dis
placed, the energy gap between the ground and the excited state will vary linearly
 with the nuclear coordinate q. A uctuating coordinate due to thermal excitation of
 the nuclear coordinates hence will give rise to a uctuating transition frequency. In
 that sense the Brownian oscillator model is a more microscopic model, which speci¯es
 the reason for a uctuating transition frequency \cite{hamm2005principlesnonlinearoptical} page 58
}}, different molecules oscillate at slightly different frequencies due to variations in their local environment. The resulting dephasing of individual oscillators leads to the characteristic decay of macroscopic coherences observed in techniques such as photon echo spectroscopy \cite{mukamel1995principlesnonlinearoptical}.



\noindent
Now that we have introduced the most important quantity in spectroscopy Eq. \ref{eq:macroscopic_polarization}, we can discuss how it depends on the incident electric field. This leads to linear and nonlinear optics.

%----------------------------------------------------------------------------------------
%	SECTION 2: NONLINEAR OPTICS
%----------------------------------------------------------------------------------------
\section{Nonlinear Optics}
\label{sec:nonlinear_optics}
\todoidea{typically at high field intensities, the response of the medium becomes nonlinear.}
\todoidea{explain Response theory as the standard way to deal with this -> it is a perturbative approach -> mine is numerical and better cite leszczynskietal2017handbookcomputationalchemistry, the most modern one is loaizaetal2025nonlinearspectroscopygeneralized}


In general, the polarization can be expressed as a power series in the electric field:

\begin{equation}
	\vec{P} = \varepsilon_0 (\chi^{(1)} \vec{E} + \chi^{(2)} \vec{E} \cdot \vec{E} + \chi^{(3)} \vec{E} \cdot \vec{E} \cdot \vec{E} + \ldots)
	\label{eq:nonlinear_polarization}
\end{equation}

\noindent 
where $\varepsilon_0$ is the vacuum permittivity and $\chi^{(i)}$ is the $i$-th-order nonlinear susceptibility tensors. 
In the linear regime, the induced polarization $\vec{P}$ can be cut after the first contribution, which is directly proportional to the applied electric field $\vec{E}$.
This relationship describes phenomena such as refraction and absorption.
The higher-order terms give rise to a variety of nonlinear optical phenomena \cite{boyd2008contents}.

\noindent 
\todoidea{this is not so important for this thesis, so keep it short:}
Second-order nonlinear processes, governed by the $\chi^{(2)}$ term, occur only in noncentrosymmetric materials, which lack inversion symmetry \cite{rao2018overviewsecondthird,boyd2008chapter1nonlinear}.  Second-order nonlinear processes include Second Harmonic Generation (SHG), where two photons of frequency $\omega$ combine to generate a photon of frequency $2\omega$; Sum Frequency Generation (SFG), where two photons of frequencies $\omega_1$ and $\omega_2$ combine to produce a photon of frequency $\omega_1 + \omega_2$; Difference Frequency Generation (DFG), where two photons interact to create a photon of frequency $\omega_1 - \omega_2$.


\subsection{Third-Order Nonlinear Processes}
\label{subsec:third_order}
\todoidea{change the order of phase matching <-> third order nonlinear process}
\todoidea{Add intuition for what the pulse phase kick does for delta pulses -> visual representation on bloch sphere like in \cite{grolletal2025fundamentalsheterodynewave} also the pulse area!! }
\noindent
\noindent
To demonstrate the complexity of nonlinear spectroscopy, that in the third order component of the polarization Eq. \eqref{eq:nonlinear_polarization}

\begin{equation}
	\vec{P}^{(3)} = \varepsilon_0 \chi^{(3)} \vec{E} \cdot \vec{E} \cdot \vec{E},
	\label{eq:third_order_polarization}
\end{equation}

each field $\vec{E}$ generally contains three field-interactions with different frequencies

\begin{equation}
\vec{E} = \vec{E}_1 e^{-i \omega_1 t} + \vec{E}_2 e^{-i \omega_2 t} + \vec{E}_3 e^{-i \omega_3 t} + \text{c.c.}.
	\label{eq:explicit_e_field_third_order}
\end{equation}

\noindent
This means already the interacting field contains $?? = 44$ terms, each oscillating at different frequencies: 
 
\begin{align}
	\pm \omega_1, 
	\pm \omega_2, 
	\pm \omega_3, 
	\pm 3 \omega_1, 
	\pm 3 \omega_2, 
	\pm 3 \omega_3, \\
	\pm (\omega_1 + \omega_2 + \omega_3), 
	\pm (\omega_2 + \omega_3 - \omega_1), 
	\pm (\omega_1 - \omega_2 + \omega_3),\\
	\pm (2 \omega_1 \pm \omega_2), 
	\pm (2 \omega_1 \pm \omega_3), 
	\pm (2 \omega_2 \pm \omega_1), 
	\pm (2 \omega_2 \pm \omega_3), 
	\pm (2 \omega_3 \pm \omega_2).
\end{align}


Third-order, nonlinear processes can occur in all materials, regardless of symmetry \cite{hamm2005principlesnonlinearoptical}. Key third-order nonlinear processes include Third Harmonic Generation (THG), where three photons combine to generate a photon of tripled frequency; Nonlinear Refraction, where the refractive index depends on light intensity, known as the Kerr effect; and Two-Photon Absorption (TPA), which involves the simultaneous absorption of two photons to excite a transition.

\noindent
The most important third-order process for this thesis is Four-Wave Mixing (FWM) \cite{boyd2008chapter6nonlinear}. Here, three incident and one generated photon interact in a way that satisfies energy conservation

\begin{equation}
	\mathbf{k}_S = \mathbf{k}_1 + \mathbf{k}_2 + \mathbf{k}_3,
	\label{eq:four_wave_mixing}
\end{equation}

\noindent
generating multiple possible pathways, including photon echo signals and allowing for 2D spectroscopy. This will be discussed in detail in the section \ref{sec:phase_matching}.

\noindent
Now that we have seen of how difficult the nonlinear optics can get, we will have to find a way to simplify the allowed interactions. The standard way to calculate the nonlinear response is to use response function formalism, which will be discussed in the next section.

\section{Light matter interaction - Calculation of the Polarization}
\label{sec:response_functions}

\noindent
The startpoint for this will be the semiclassical Hamiltonian of a quantum mechanically modeled matter-system interacting with classical radiation field described with vector and scalar potentials \todofix{fix this sentence (and) or add the full Hamiltonian}.
Standard textbooks like \cite{garrisonchiao2008quantumoptics,gerryknight2024introductoryquantumoptics} go into the dipole approximation, where the Hamiltonian takes the form

\begin{equation}
	H = H_0 - \mathbf{\hat{\mu}} \cdot \mathbf{E}(t).
\end{equation}

\noindent
Here we remember the electrical field from Eq. \ref{eq:plane_wave} and $\mathbf{\hat{\mu}}$ is the dipole operator of the matter system. This allows us to calculate the macroscopic polarization as the expectation value of the dipole operator

\begin{equation}
	\vec{P}(t) = \langle \mathbf{\hat{\mu}}(t) \rangle = \mathrm{Tr}[\mathbf{\hat{\mu}} \rho(t)]
	\label{eq:polarization_expectation_value}
\end{equation}

\noindent
Similar to the electric field in Eq. \eqref{eq:plane_wave} also the polarization can be decomposed into positive and negative frequency components. For the spectroscopic measurements the positive frequency part is relevant as the emitted signal field is proportional to it \todoref{IS THIS TRUE??} \cite{mukamel1995principlesnonlinearoptical}


\noindent
Now to isolate the effect of a third order non-linear process, one can write Eq. \eqref{eq:polarization_expectation_value} in a perturbative series expansion

\begin{equation}
	\vec{P}(t) = \sum_n \vec{P}^{(n)}(t) = \sum_n \langle \mathbf{\hat{\mu}}(t) \rangle = \mathrm{Tr}[\mathbf{\hat{\mu}} \rho^{(n)}(t)],
	\label{eq:polarization_expectation_value_perturbative}
\end{equation}


\noindent
where only the quantity $\vec{P}^{(n)}(t)$ for that specific order has to be calculated.
This leads to the framework of Response functions. 

\noindent
The following equations are taken for the optical case of taking the dipole operator as the observable $A = \hat{\mu}$, with the dipole approximation Hamiltonian, but generally they hold for other observable as well, just producing different expectation values. \todoref{<- find examples?}
The $i$-th order polarization can be written as the convolution of $i$ electric fields \cite{hamm2005principlesnonlinearoptical}

\begin{align}
	P^{(i)}(t) &= \int_{0}^{\infty}\! dt_{i} \int_{0}^{\infty}\! dt_{i-1} \cdots \int_{0}^{\infty}\! dt_{1} \notag \\
	&\quad E\bigl(t-t_{i}\bigr)\, E\bigl(t-t_{i}-t_{i-1}\bigr)\,\cdots\, E\bigl(t-t_{i}-\cdots - t_{1}\bigr) \notag \\
	&\quad R(t_{i},t_{i-1},\ldots,t_{1}),
	\label{eq:response_convolution}
\end{align}

\noindent
where the $i$-th response function is defined as

\begin{equation}
	R^{(i)}(t_{i},\ldots,t_{1})
	\;=\;
	\left(-\frac{i}{\hbar}\right)^{i}
	\Big\langle
	\hat{\mu}(t_{i}+\cdots + t_{1})\;[\hat{\mu}(t_{i-1}+\cdots + t_{1}),\ldots [\hat{\mu}(0),\rho(-\infty)]\ldots]
	\Big\rangle
	\label{eq:response_function_R}
\end{equation}

\noindent
and describes the system's response to $i$ interactions with the electric field.
Here $\hat{\mu}(t)$ is the dipole operator in the interaction picture.
Note, that the interactions at the different times $t - t_{i} - \ldots$ can occur in any order.
Also note, that the response function only depends on the material under investigation \cite{hamm2005principlesnonlinearoptical}.
It is assumed that the interaction takes place at local times $t_i$ (short pulse limit), and the system is in an equilibrium state (the ground state) before the interaction, at time $t = - \infty$.
It is crucial that the first $i-1$ dipole operators are inside nested commutators, that represent how these field-interactions drive the system out of equilibrium. The last operator then represents the emitted field and is measured by being traced over the environmental degrees of freedom.
Explicitly, the commutators in Eq-\eqref{eq:response_function_R} contains $2^i$ terms with multiple actions of $\hat{\mu}$ on the left / right on the density matrix.
Thought one can ignore the complex conjugate contributions, reducing to $2^{i-1}$ terms.
Each such term corresponds to a specific sequence of field interactions and can be represented with double-sided Feynman diagrams \cite{mukamel1995principlesnonlinearoptical}. In it the left and right sides of the diagram represent the evolution of the ket and bra states of the density matrix, respectively. Each interaction with the electric field is depicted as an arrow, with arrows pointing towards the density matrix indicating absorption (interaction from the left) and arrows pointing away indicating emission (interaction from the right). The time intervals between interactions are represented by horizontal lines, during which the system evolves freely according to its Hamiltonian.
\todoidea{add a example figure for the simplest linear polarizataion and then later the pathways for the third order response function.}


\noindent
Eq. \eqref{eq:response_convolution} is the central equation in two dimensional spectroscopy \cite{segarra-martietal2018accuratesimulationtwodimensional}. Each experimentally observed signal direction (phase matched condition Eq. \eqref{eq:phase_matching}) isolates a subset of the possible Liouville-space pathways of the system.
Accordingly, the total response in that direction can be written as a sum over all contributing pathways,
\begin{equation}
	R^{(i)}(t) = \sum_{\text{path}} R_{\text{path}}(t),
	\label{eq:Liouville_pathways}
\end{equation}
where $R_{\text{path}}(t)$ represents the contribution of a single Liouville pathway corresponding to a specific sequence of field interactions and density-matrix evolutions.


\noindent
As mentioned in Sec \ref{subsec:third_order}, FWM is the method discussed in detail. It is a third-order process, so $i=3$ in Eq. \eqref{eq:response_convolution} and \eqref{eq:response_function_R}. The response function can be separated into different contributions, depending on the time ordering of the three field interactions. This leads to four main contributions, which can be represented with double-sided Feynman diagrams \cite{mukamel1995principlesnonlinearoptical}.


The setup can be seen in Fig. \ref{fig:fwm_box_car_setup}. Here three pulses with wavevectors $\mathbf{k}_1, \mathbf{k}_2, \mathbf{k}_3$ interact with the sample and generate a signal in the direction $\mathbf{k}_S$. 

\noindent
This thesis will instead use a more microscopic non-perturbative (NP) approach.
Here first the total signal will be calculated, where all pulses are active, and then the desired order will be extracted by filtering out the linear contributions, where only the $i$-th pulse is active.
This way the third order signal can be calculated by 

\begin{equation}
	P^{(3)}(t) = P_{\text{total}}(t) - P_{(2)}(t) - P_{(1)}(t) - P_{(0)}(t)
	\label{eq:third_order_signal_numerical}
\end{equation}

\noindent
where the Polarization from Eq. \eqref{eq:polarization_expectation_value} is used. \todofix{actually not true as only the analytical component is extracted, to include phase information.}



% TODO uncomment these figures
\todofix{uncomment figures (for git -> also include svgs?)}
\iffalse
\begin{figure}[ht]
	\centering
	\includegraphics[width=0.8\textwidth]{../figures/FWM_scheme.pdf}
	\caption{Schematic Four-wave mixing setup in a Boxcar geometry. ... \todoimp{Add description. Add S, Add wavevectors, Reduce amplitude of probe pulse.}}
	\label{fig:fwm_box_car_setup}
\end{figure}

\begin{figure}[ht]
	\centering
	\includegraphics[width=0.8\textwidth]{../figures/FWM_scheme_phase_cycling.pdf}
	\caption{Schematic Four-wave mixing setup. This is the simpler collinear phase-cycling setup, at the cost of requiring multiple measurements with varied pulse phases.
	\todoidea{add ground state at beginning, }}
	\label{fig:fwm_phase_cycling_setup}
\end{figure}
\fi

\todoidea{Interaction picture, rotating-wave approximation (introduce but defer justification).}
\todoidea{Also include how to simulate laser pulses:
Gaussian envelope, pulse trains.
1) Time-domain field (carrier × envelope)
$f(t)$ = envelope (e.g. Gaussian $f(t) = \exp\left[-\frac{2\ln 2}{\tau_{\mathrm{FWHM}}^2} t^2\right]$),

$\phi_{\mathrm{CE}}$ = carrier--envelope phase (relevant for few-cycle pulses),

$E_0$ sets peak field (related to pulse energy via the mode area and duration).

Bandwidth for a transform-limited Gaussian:

$\Delta\omega_{\mathrm{FWHM}} \approx \frac{2\ln 2}{\tau_{\mathrm{FWHM}}}$}






\subsection{Heterodyne and Homodyne Detection}
\label{subsec:heterodyne_homodyne}

\noindent There are two main detection schemes used, influencing sensitivity and the information content of the measurements: homodyne and heterodyne detection \cite{abramaviciusetal2009coherentmultidimensionaloptical}.

\noindent
Generally the signal that arrives at the detector is proportional to the intensity of the emitted electric field, which is then integrated over a period of time by the detector

\begin{equation}
	S_{\text{det}} \propto \int I_{\text{det}}(t) \, dt.
	\label{eq:signal_intensity}
\end{equation}

\noindent On the one hand in homodyne detection, the intensity of the emitted signal field is measured directly, so $ I_{\text{HOD}} \propto |E_{\text{S}}|^2 \propto |P^{(3)}|^2 $. While experimentally simpler, homodyne detection loses phase information and it is more susceptible to noise at low signal levels \todoref{find ref or delete}%\cite{abramaviciusetal2009coherentmultidimensionaloptical}.

\noindent On the other hand, heterodyne detection involves the interference of the signal field with a reference field, the so called local oscillator (LO) of known amplitude and phase:

\begin{equation}
	I_{\text{HET}} \propto |E_{\text{S}} + E_{\text{LO}}|^2 \approx 2\Re{(E_{\text{LO}}^*E_{\text{S}})} + |E_{\text{LO}}|^2 + |E_{\text{S}}|^2
	\label{eq:heterodyne}
\end{equation}

\noindent Since $|E_{\text{LO}}| \gg |E_{\text{S}}|$, the cross-term dominates, and the signal can be extracted by phase-cycling. In contrast to homodyne detection, the heterodyned signal is not background-free but sits on an offset $\int E_{\text{LO}}(t)^2 \, dt$ which can be subtracted. Ideally, shaped pulses are used for the three input pulses and the local oscillator, yielding the response function directly: $\int E_{\text{LO}}(t) P^{(3)}(t) \, dt = S^{(3)}(t_3, t_2, t_1)$, without convolution complications, where $t_1$, $t_2$, $t_3$ are the directly controlled delays. This is exactly the analytically defined third-order response function in Eq. \eqref{eq:response_function_R}, making experimental data and simulations directly interpretable. This is the main strength of heterodyne detection \cite{mukamel1995principlesnonlinearoptical}. Also, it provides both amplitude and phase information of the signal. The signal scales linearly with the third-order response \todoref{find ref} \todoidea{why is this good?}

\noindent In modern multidimensional spectroscopy, spectral interferometry—a form of heterodyne detection—is widely employed \cite{hybletal1998twodimensionalelectronicspectroscopy}. The signal field and a time-delayed reference pulse are spatially overlapped and spectrally resolved.

\section{Wavevector Phase-Matching Conditions}
\label{sec:phase_matching}

\noindent 
In nonlinear optical processes, multiple light waves interact within a material to generate new frequencies. For these processes to be efficient, the phase relationship between the interacting waves must be maintained throughout the propagation distance. This condition is known as phase-matching.
\noindent 
Like mentioned before, it represents momentum conservation of the incident and outgoing photons. For a general nonlinear process, the wavevector of the generated signal ($\vec{k}_s$) is determined by the vector sum of the input wavevectors:

\begin{equation}
	\vec{k}_s = \pm\vec{k}_1 \pm\vec{k}_2 \pm\vec{k}_3 \pm \ldots
	\label{eq:phase_matching}
\end{equation}

\noindent 
The signs depend on whether the corresponding field acts as a "bra" ($-$) or a "ket" ($+$) in the quantum mechanical description, which corresponds to photon emission or absorption, respectively.

Each laser pulse interacts with the dipole operator $\mu$.
In the rotating-wave approximation (RWA), each field $E_i(t) \propto e^{-i\omega_i t}$ can either:
\todoimp{fact check this:}
Excite the ket side: $\mu_{-} E_i(t) \rightarrow e^{-i\omega_i t} \rightarrow$ forward coherence $e^{-i\omega_{eg} t}$

De-excite the bra side: $\mu_{+} E_i^{*}(t) \rightarrow e^{+i\omega_i t} \rightarrow$ backward (complex-conjugate) coherence $e^{+i\omega_{eg} t}$

That's why the sign in the phase-matching condition ($\pm k_i$) tells you which side of the density matrix the field acted on.

\noindent 
Different combinations of signs correspond to different phase-matching conditions, leading to signals in different spatial directions. These distinct signal directions allow separation of various nonlinear optical processes.


\subsection{Rephasing and nonrephasing Signals}
\label{subsec:rephasing_nonrephasing}

\noindent 
To reduce the number of contributions to the third-order polarization, you can make three vital simplifications: 1.) strict time ordering of the pulses, 2.) the rotating wave approximation (RWA), and 3.) phase-matching \cite{hamm2005principlesnonlinearoptical,  % TODO delete the rest
mukamel1995principlesnonlinearoptical, cho2009twodimensionalopticalspectroscopy, jonas2003twodimensionalfemtosecondspectroscopy}.
\todoidea{<- TODO: mention impulsive limit / delta pulses \cite{cho2009twodimensionalopticalspectroscopy}}

For these assumptions, the only surviving terms in the third order polarization are the rephasing and nonrephasing contributions \cite{cho2009twodimensionalopticalspectroscopy, jonas2003twodimensionalfemtosecondspectroscopy}.
\textbf{Rephasing signal} follow the phase-matching condition $\vec{k}_s = -\vec{k}_1 + \vec{k}_2 + \vec{k}_3$. The first interaction creates a coherent superposition between ground and excited states, generated by a complex conjugate interaction of the "bra". The second pulse converts this coherence into a population, and the third pulse generates a new coherence that emits the signal. The phase accumulated due to different frequencies can be reversed during the period between the third interaction and signal emission. This leads to a photon echo effect at $t_{\text{det}} \approx t_{\text{coh}}$ where the dephased components rephase.
This happens for an inhomogeneously broadened ensemble of systems, where different realizations of the system have different transition energies, while the laser pulses drive at constant frequency of $\omega_L$. As a result, some realizations dephase quicker than others. Some realizations should come back in phase around $t_{\text{det}} \approx t_{\text{coh}}$ which is exactly this photon echo.
Contrary to this, \textbf{nonrephasing signal} satisfy $\vec{k}_s = +\vec{k}_1 - \vec{k}_2 + \vec{k}_3$. In this case, the first pulse creates a forward-evolving coherence, and the phase evolution at the third pulse continues in the same direction, and no echo is formed.


\subsection{Phase Cycling in Nonlinear Spectroscopy}
\label{subsec:phase_cycling}

\noindent 
While phase-matching enables the spatial isolation of desired nonlinear Liouville pathways, phase cycling provides an alternative approach by distinguishing signals based on their phases rather than their spatial propagation directions. \todoref{find ref}%\cite{tan2008, yan2009}.
\cite{mukamel1995principlesnonlinearoptical, cho2009twodimensionalopticalspectroscopy, jonas2003twodimensionalfemtosecondspectroscopy, brixneretal2004phasestabilizedtwodimensionalelectronic, greenetal2024vibrationalcoherenceshalfbroadband}.

\noindent 
Here the collinear beam line geometry can be used, simplifying the experimental setup.
This technique is particularly valuable when studying samples without scattering.

\noindent 
And for simulations, as done here, it facilitates the separation of rephasing and nonrephasing signals within the same measurement series.

\noindent 
In phase cycling, multiple measurements are taken with systematically varied phases of the excitation pulses. The desired nonlinear signals can then be extracted through appropriate linear combinations of these measurements. The fundamental principle relies on the fact that different nonlinear pathways respond distinctly to changes in the phases of the input fields.

\noindent 
For a third-order signal generated by three excitation pulses with phases $\phi_1$, $\phi_2$, and $\phi_3$, as seen in Fig. \ref{fig:fwm_phase_cycling_setup}.

\noindent 
By varying the input phases through a complete cycle (in steps of $\pi/2$) and applying discrete Fourier transform to the collected data, specific pathways can be isolated based on their phase dependencies.


\subsection{Phase-Cycling Fourier Selection and Construction of 2D Spectra}
\label{subsec:phase_cycling_fourier_selection}


\noindent 
The nonlinear polarization induced by three incident pulses can be expressed as a Fourier series in the pulse phases $\phi_1, \phi_2, \phi_3$, and individual phase-matched components are isolated via an inverse Fourier transform:

\begin{align}
	P^{(3)}_{n_1,n_2,n_3}(t_{\text{coh}},T,t_{\text{det}}) =
	\frac{1}{(2\pi)^3} \int_{0}^{2\pi}	\int_{0}^{2\pi} \int_{0}^{2\pi}  
	& d\phi_1 d\phi_2 d\phi_3
	e^{-i(n_1\phi_1+n_2\phi_2+n_3\phi_3)} \\
	& P^{(3)}(\phi_1,\phi_2,\phi_3;t_{\text{coh}},T,t_{\text{det}}).
	\label{eq:continuous_phase_cycling}
\end{align}

\noindent 
The physically relevant phase-matching directions correspond to:

\begin{align}
	\vec{k}_{\mathrm{R}}  & = -\vec{k}_1 + \vec{k}_2 + \vec{k}_3,
	                      & (n_1,n_2,n_3)                         & = (-1,+1,+1), \label{eq:rephasing_selection}    \\
	\vec{k}_{\mathrm{NR}} & = +\vec{k}_1 - \vec{k}_2 + \vec{k}_3,
	                      & (n_1,n_2,n_3)                         & = (+1,-1,+1), \label{eq:nonrephasing_selection}
\end{align}

\noindent 
yielding the rephasing (R) and nonrephasing (NR) contributions, respectively.

\noindent 
The (wavevector) phase-matched rephasing signal direction corresponds to selecting the Fourier index triple $(n_1,n_2,n_3)=(-1,+1,+1)$, while the nonrephasing direction corresponds to $(+1,-1,+1)$ \cite{mukamel1995principlesnonlinearoptical, cho2009twodimensionalopticalspectroscopy, greenetal2024vibrationalcoherenceshalfbroadband}.

\noindent 
The emitted field in a given phase-matching direction is proportional to the corresponding polarization\cite{jonas2003twodimensionalfemtosecondspectroscopy}:

\begin{equation}
	E_{k_s}(t_{\text{coh}},T,t_{\text{det}}) \propto i P_{\vec{k}_S}(t_{\text{coh}},T,t_{\text{det}}).
	\label{eq:field_polarization_relation}
\end{equation}

This holds in the slowly varying amplitude approximation \cite{mukamel1995principlesnonlinearoptical}.


\paragraph{Time--Frequency Transforms.}

\noindent 
So again i want to emphasize that first, dime domain signals of the third order polarization are calculated using Eq. \eqref{eq:polarization_expectation_value}, Eq. \eqref{eq:third_order_signal_numerical} and then the desired phase-matched contributions are extracted using Eq. \eqref{eq:continuous_phase_cycling}. 
After phase selection, the third order complex time-domain field $E_{k_S}(t_{\text{coh}}, T, t_{\text{det}})$ is Fourier transformed over the detection time $t_{\text{det}}$ and the coherence time $t_{\text{coh}}$ to construct the two-dimensional spectra (cf. Eq.~\eqref{eq:2des_signal}):

\begin{align}
	S_{R}(\omega_{\text{coh}}, T, \omega_{\text{det}})
	 & =
	\int dt_{\text{coh}} \int dt_{\text{det}} \;
	e^{+ i \omega_{\text{coh}} t_{\text{coh}}} e^{- i \omega_{\text{det}} t_{\text{det}}}
	E_{R}(t_{\text{coh}}, T, t_{\text{det}}),
	\label{eq:rephasing_transform} \\
	S_{NR}(\omega_{\text{coh}}, T, \omega_{\text{det}})
	 & =
	\int dt_{\text{coh}} \int dt_{\text{det}} \;
	e^{- i \omega_{\text{coh}} t_{\text{coh}}} e^{- i \omega_{\text{det}} t_{\text{det}}}
	E_{NR}(t_{\text{coh}}, T, t_{\text{det}}).
	\label{eq:nonrephasing_transform}
\end{align}

\noindent 
The opposite sign in the $t_{\text{coh}}$-exponent implements the standard convention distinguishing rephasing (photon-echo) and nonrephasing contributions \cite{cho2009twodimensionalopticalspectroscopy, greenetal2024vibrationalcoherenceshalfbroadband}. In discrete numerical implementations, Eqs.~\eqref{eq:rephasing_transform}--\eqref{eq:nonrephasing_transform} are approximated by (shifted) fast Fourier transforms \cite{cho2009twodimensionalopticalspectroscopy, greenetal2024vibrationalcoherenceshalfbroadband}.

\paragraph{Absorptive Combination.}

\noindent 
The experimentally measured (or simulated) complex spectra $S_{R}$ and $S_{NR}$ can be combined to  construct the purely absorptive 2D spectrum \cite{mukamel1995principlesnonlinearoptical, jonas2003twodimensionalfemtosecondspectroscopy, greenetal2024vibrationalcoherenceshalfbroadband}
\begin{equation}
	S_{\text{abs}}(\omega_{\text{coh}}, T, \omega_{\text{det}})
	=
	\Re \left\{
	S_{R}(\omega_{\text{coh}}, T, \omega_{\text{det}}) + S_{NR}(\omega_{\text{coh}}, T, \omega_{\text{det}})
	\right\}.
	\label{eq:absorptive_spectrum}
\end{equation}

\noindent 
The absorptive spectra is free from broadening refractive contributions \cite{fullerogilvie2015experimentalimplementationstwodimensional}.



%----------------------------------------------------------------------------------------
%	SECTION 4: PHOTON ECHO SPECTROSCOPY
%----------------------------------------------------------------------------------------
\section{Photon Echo Spectroscopy}
\label{sec:photon_echo}
\noindent 
The echo intensity as a function of the delay time $t_{\text{coh}}$ reveals information about the dephasing processes in the system.

\noindent 
It is important to emphasize that static inhomogeneity is crucial for the appearance of the photon echo signal. This inhomogeneity arises from variations in the local environment of individual quantum systems within the ensemble, leading to a distribution of transition frequencies. In this thesis, in the numerical simulations, static inhomogeneity is taken into account by averaging the results over an ensemble of different realizations of the Hamiltonian \cite{cho2009twodimensionalopticalspectroscopy, mukamel1995principlesnonlinearoptical}. Without this inhomogeneous distribution, all systems would evolve identically, and the characteristic echo phenomenon would not occur.

\noindent 
By scanning the waiting time $T$, this technique allows measurement of population dynamics and spectral diffusion processes.

\subsection{Two-Dimensional Electronic Spectroscopy}
\label{subsec:2d_spectroscopy}

\noindent 
It correlates excitation and detection frequencies, revealing couplings between different transitions and energy transfer pathways.

\noindent 
Recall that a complete 2D spectrum is built from repeated three-pulse sequences while the coherence time $t_{\text{coh}}$ is scanned. The rephasing photon-echo signal arises after a finite rephasing time $t$ for sequences in which pulse 1 precedes pulse 2 ($t_{\text{coh}}>0$) and is recorded in the phase-matched direction $-\vec{k}_1 + \vec{k}_2 + \vec{k}_3$. As discussed in Subsec.~\ref{subsec:rephasing_nonrephasing}, the ground–excited coherences during $t_{\text{coh}}$ and $t$ accumulate opposite phases in the rephasing branch. When the ordering of pulses 1 and 2 is reversed ($t_{\text{coh}}<0$), these phase factors add rather than cancel, yielding a nonrephasing free-induction–decay signal that emerges immediately following pulse 3; this contribution provides information complementary to the rephasing branch \cite{ginsbergetal2009twodimensionalelectronicspectroscopy}.

\noindent The resulting 2D spectrum contains peaks along the diagonal ($\omega_{\text{coh}} = \omega_{\text{det}}$) corresponding to the linear absorption spectrum, while off-diagonal peaks reveal couplings and energy transfer between different states. The evolution of the 2D spectra with waiting time $T$ provides detailed information about energy transfer kinetics, spectral diffusion, and quantum coherence effects.

\noindent In this sense, 2D spectroscopy is the ultimate nonlinear experiment within third-order spectroscopy, gathering the maximum information—what cannot be learned with 2D spectroscopy cannot be learned with any other third-order method. However, the response function $S^{(3)}(t_3, t_2, t_1)$ is a three-dimensional oscillating function that is complex and difficult to visualize. This is why heterodyne-detected photon echoes, pioneered by Wiersma in the mid-90s, were initially not considered useful.

\subsection{Applications of Photon Echo Spectroscopy}
\label{subsec:echo_applications}

\noindent Photon echo techniques have been applied to a wide range of problems across chemistry, biology, and materials science:

\textbf{Exciton dynamics} in photosynthetic complexes, revealing quantum coherent energy transfer pathways \todoref{find ref}%\cite{engel2007, schlau-cohen2011}
\textbf{Vibrational dynamics} in proteins and liquids, elucidating structural fluctuations and hydrogen-bonding networks \cite{hammzanni2011conceptsmethods2d}
\textbf{Charge transfer processes} in organic photovoltaics and light-harvesting systems
\textbf{Coupling mechanisms} between electronic and vibrational degrees of freedom \cite{khaliletal2004vibrationalcoherencetransfer}



%----------------------------------------------------------------------------------------
\todoidea{
Focus on electronic spectroscopy (e.g., dephasing, dissipation), and biological contexts (e.g., microtubules as complex open systems). 

Add Thesis-Specific Emphasis: Throughout, weave in connections to open quantum systems (e.g., in "Energy Dissipation," link to Lindblad operators or bath interactions). In "Two-Dimensional Electronic Spectroscopy," explicitly state how 2DES reveals couplings and dynamics in systems like microtubules. Add a forward-looking paragraph in the intro/outro noting how this chapter sets up your modeling work.

Enhance Biological Relevance: In "Applications," expand on exciton dynamics in photosynthetic systems as analogs to microtubule coherence. Add a subsection or paragraph on "Towards Biological Systems" at the end, discussing how 2DES can probe quantum effects in cytoskeletal structures.

Improve Flow and Language: Add transitional sentences (e.g., "Building on these fundamentals, we now explore nonlinear optics..."). Use active voice and concise language. Fix minor issues like inconsistent capitalization (e.g., "Rephasing" vs. "rephasing").

Add Figures and Examples
}

\todoidea{real - absorption and imaginary - refractive part of the 2d spectrum, first available after LO heterodyne detection \cite{jonas2003twodimensionalfemtosecondspectroscopy}

In terms of coherence pathways, the T = 0 experiment is the optical analog
of the simplest COSY 2D NMR Fourier transform correlation experiment (see
figure 6.3.1 of Reference 2
-> WHAT IS THIS??

Because 2D FT spectra are additive, the narrow cross-width of the spectrum
perpendicular to the diagonal suggested the 2D FT experiment resolved the “ho-
mogeneous lineshape” for an instantaneous local solvent environment around each
dye molecule (5). 
[homogeneous - antidiagonal, inhomogeneous - diagonal broadening]
}

% Chapter Template

\chapter{Example Two-Level Atom System} % Main chapter title

\label{chapter_rabi_oscillations} % For referencing this chapter elsewhere, use \ref{chapter_rabi_oscillations}

%----------------------------------------------------------------------------------------
%	SECTION: Two-Level Atom Master Equation in the Rotating Frame with RWA
%----------------------------------------------------------------------------------------

\section{Master Equation in the Rotating Frame with RWA}

Rabi oscillations describe the coherent oscillatory behavior of a two-level quantum system interacting with a resonant electromagnetic field.
This phenomenon is fundamental in quantum mechanics and quantum optics.


\subsection{Hamiltonian in the Laboratory Frame}
Consider a two-level system with states \(|g\rangle\) (ground state) and \(|e\rangle\) (excited state).
The energy separation between the two states is given by:
\begin{equation}
	\omega_0 = \frac{E_e - E_g}{\hbar},
	\label{eq:EnergySeparation}
\end{equation}
where \(E_e\) and \(E_g\) are the energies of the excited and ground states, respectively.

\subsubsection{Atomic Hamiltonian}
The free atomic Hamiltonian is expressed as:
\begin{equation}
	H_0 = \frac{\hbar\omega_0}{2} \sigma_z, \qquad \sigma_z = \ket{e}\bra{e} - \ket{g}\bra{g}
	\label{eq:AtomicHamiltonian}
\end{equation}

\subsubsection{Electric Field}
The electric field is given analogously to Eq. \eqref{eq:electric_field_three_pulses} by:
\begin{equation}
	\mathcal{E}(t) = E_0 e^{i\phi} e^{-i\omega_L t}, \qquad
	E_{\text{phys}}(t) = \mathcal{E}(t) + \mathcal{E}^*(t)
	\label{eq:ElectricFieldComplex}
\end{equation}
where \(E_0\) is the field amplitude, \(\phi\) is the phase, and \(\omega_L\) is the laser frequency.

\subsubsection{Interaction Hamiltonian}
Using the electric dipole interaction with raising and lowering operators \(\sigma_\pm = (\sigma_x \pm i\sigma_y)/2\):
\begin{equation}
	H_{\text{int}}(t) = -\mu \left( \mathcal{E}(t) \sigma_+ + \mathcal{E}^*(t) \sigma_- \right)
	\label{eq:InteractionHamiltonianDipole}
\end{equation}
where \(\mu\) is the dipole matrix element.

\subsubsection{Total Hamiltonian}
The total Hamiltonian of the system is:
\begin{equation}
	H(t) = H_0 + H_{\text{int}}(t)
	\label{eq:TotalHamiltonian}
\end{equation}

The time evolution of the system is governed by the time-dependent Schrödinger equation \eqref{eq:SchrödingerEquation}

\subsection{Transform to the Rotating Frame}

Define the unitary transformation:
\begin{equation}
	U(t) = \exp\left(i \omega_L t \frac{\sigma_z}{2}\right)
	\label{eq:UnitaryRotatingFrame}
\end{equation}

Transform operators as \(\tilde{X} = U X U^\dagger\), and the density matrix as \(\tilde{\rho} = U \rho U^\dagger\).

Useful identities for the transformation:
\begin{equation}
	U \sigma_\pm U^\dagger = e^{\pm i \omega_L t} \sigma_\pm, \qquad
	U \sigma_z U^\dagger = \sigma_z
	\label{eq:TransformationIdentities}
\end{equation}

\subsubsection{Transformed Hamiltonian Terms}

\begin{itemize}
	\item \textbf{Free Hamiltonian:}
	      \begin{equation}
		      \tilde{H}_0 = \frac{\hbar(\omega_0 - \omega_L)}{2} \sigma_z = \frac{\hbar\Delta}{2} \sigma_z
		      \label{eq:TransformedFreeHamiltonian}
	      \end{equation}
	      where \(\Delta = \omega_0 - \omega_L\) is the detuning.

	\item \textbf{Transformed interaction Hamiltonian:}
	      \begin{equation}
		      \tilde{H}_{\text{int}} = -\mu \left( E_0 e^{i\phi} \sigma_+ + E_0 e^{-i\phi} \sigma_- \right)
		      \label{eq:TransformedInteractionHamiltonian}
	      \end{equation}

	\item Defining the Rabi frequency \(\Omega = \mu E_0 / \hbar\), the total rotating-frame Hamiltonian becomes:
	      \begin{equation}
		      \boxed{
			      \tilde{H} = \frac{\hbar\Delta}{2} \sigma_z - \hbar \Omega \left( e^{i\phi} \sigma_+ + e^{-i\phi} \sigma_- \right)
		      }
		      \label{eq:RotatingFrameHamiltonian}
	      \end{equation}
\end{itemize}

\subsection{Master Equation in the Rotating Frame (with RWA)}

The master equation in the rotating frame is:
\begin{equation}
	\dot{\rho} = -i[\tilde{H}, \rho] + \mathcal{D}(\rho)
	\label{eq:MasterEquationRotatingFrame}
\end{equation}

The dissipator includes spontaneous emission and pure dephasing:
\begin{equation}
	\mathcal{D}(\rho) =
	\gamma_0 \left( \sigma_- \rho \sigma_+ - \frac{1}{2} \{ \sigma_+ \sigma_-, \rho \} \right)
	+ \frac{\gamma_\varphi}{2} \left( \sigma_z \rho \sigma_z - \rho \right)
	\label{eq:DissipatorOperator}
\end{equation}

Define the total decoherence rate:
\begin{equation}
	\Gamma = \frac{\gamma_0}{2} + \gamma_\varphi
	\label{eq:TotalDecoherenceRate}
\end{equation}

\subsection{Equations of Motion}

Write the density matrix as:
\begin{equation}
	\rho = \begin{pmatrix}
		\rho_{gg} & \rho_{ge} \\
		\rho_{eg} & \rho_{ee}
	\end{pmatrix}, \qquad \rho_{ge} = \rho_{eg}^*
	\label{eq:DensityMatrixElements}
\end{equation}

Let \(\Omega_c = \Omega e^{i\phi}\). The equations of motion are:
\begin{equation}
	\boxed{
		\begin{aligned}
			\dot{\rho}_{ee} & = i \Omega_c \rho_{ge} - i \Omega_c^* \rho_{eg} - \gamma_0 \rho_{ee}            \\[6pt]
			\dot{\rho}_{gg} & = -i \Omega_c \rho_{ge} + i \Omega_c^* \rho_{eg} + \gamma_0 \rho_{ee}           \\[6pt]
			\dot{\rho}_{eg} & = i \Delta \rho_{eg} + i \Omega_c (\rho_{gg} - \rho_{ee}) - \Gamma \rho_{eg}    \\[6pt]
			\dot{\rho}_{ge} & = -i \Delta \rho_{ge} - i \Omega_c^* (\rho_{gg} - \rho_{ee}) - \Gamma \rho_{ge}
		\end{aligned}
	}
	\label{eq:BlochEquations}
\end{equation}

On resonance (\(\Delta = 0\)), these reduce to the form implemented in the Liouvillian matrix in numerical calculations.

\subsection{Liouvillian Matrix Form}

Flatten the density matrix into a column vector:
\todoidea{
	use the column stacking version like qutip uses!
}
\begin{equation}
	\vec{\rho} =
	\begin{pmatrix}
		\rho_{gg} \\
		\rho_{ge} \\
		\rho_{eg} \\
		\rho_{ee}
	\end{pmatrix}
	\label{eq:DensityMatrixVector}
\end{equation}

Then the master equation becomes:
\begin{equation}
	\dot{\vec{\rho}} = L(t) \vec{\rho}
	\label{eq:LiouvillianMatrixForm}
\end{equation}

Where \(L(t)\) is the time-dependent Liouvillian matrix, using:
\begin{equation}
	\Omega_c(t) = \frac{\mu}{\hbar} E(t), \quad \text{with } E(t) = E_0 e^{i\phi}
	\label{eq:TimeDepRabiFreq}
\end{equation}

This matrix form is equivalent to the operator master equation above, and is suitable for numerical integration.

\subsection{Rabi Frequency and Dynamics}

The Rabi frequency \(\Omega_R\) determines the oscillation rate between the two levels:
\begin{equation}
	\Omega_R = \sqrt{\Delta^2 + \Omega^2}
	\label{eq:RabiFrequency}
\end{equation}

The phase \(\phi\) does not change the \textbf{Rabi frequency}, but it \textbf{rotates the axis of Rabi oscillations} in the Bloch sphere — i.e., it changes the \textbf{initial direction} of the drive.

In matrix form, in the basis \(\{ \ket{e}, \ket{g} \}\), the RWA Hamiltonian is:
\begin{equation}
	\tilde{H}_{\text{RWA}} = \frac{\hbar}{2}
	\begin{pmatrix}
		\Delta             & -\Omega e^{i\phi} \\
		-\Omega e^{-i\phi} & -\Delta
	\end{pmatrix}
	\label{eq:RWAHamiltonianMatrix}
\end{equation}

\subsection{Recovery of Laboratory Frame Density Matrix}

To recover the entries of the original density matrix \(\rho(t)\) from the evolved density matrix in the rotating frame \(\tilde{\rho}(t)\), we use the inverse transformation:
\begin{equation}
	\rho(t) = U^\dagger(t) \tilde{\rho}(t) U(t)
	\label{eq:RecoverOriginalDensityMatrix}
\end{equation}

The entries of the original density matrix \(\rho(t)\) are related to the entries of the density matrix in the rotating frame \(\tilde{\rho}(t)\) by:
\begin{align}
	\rho_{gg}(t) & = \tilde{\rho}_{gg}(t) \label{eq:RecoveryGG}                  \\
	\rho_{ee}(t) & = \tilde{\rho}_{ee}(t) \label{eq:RecoveryEE}                  \\
	\rho_{ge}(t) & = e^{i\omega_L t} \tilde{\rho}_{ge}(t) \label{eq:RecoveryGE}  \\
	\rho_{eg}(t) & = e^{-i\omega_L t} \tilde{\rho}_{eg}(t) \label{eq:RecoveryEG}
\end{align}

%----------------------------------------------------------------------------------------
%	SECTION: Applications and Implications
%----------------------------------------------------------------------------------------

\section{Applications and Implications}

Rabi oscillations and the associated theoretical tools, such as the RWA and density matrix formalism, have wide-ranging applications:
\begin{itemize}
	\item \textbf{Quantum Computing:} Rabi oscillations are used to implement quantum gates by precisely controlling the population of qubits.
	\item \textbf{Spectroscopy:} The Rabi frequency provides information about the interaction strength between light and matter.
	\item \textbf{Atomic Physics:} Understanding Rabi oscillations is essential for manipulating atomic states in experiments.
\end{itemize}

These concepts form the foundation for advanced topics in quantum mechanics and quantum technologies.

\include{chapters/c50_model}
%\include{chapters/todo_example}

\iffalse
	# comment out everything below this line
\fi
%-------------------------------------------------------------------------------
%	APPENDICES
%-------------------------------------------------------------------------------

\appendix

%% !TEX root = ../main.tex
% Appendix Template

\chapter{Appendix Title Here} % Main appendix title

\label{AppendixX} % Change X to a consecutive letter; for referencing this appendix elsewhere, use \ref{AppendixX}

Write your Appendix content here. % corrected appendix path
% !TEX root = ../main.tex
\chapter{Numerical Implementation}
\label{chap:numerical_implementation} % TODO put this chapter as an appendix!


\noindent
This chapter presents the computational framework developed for simulating one- and two-dimensional electronic spectroscopy of open quantum systems. The implementation is built around the \texttt{qspectro2d} Python package, which provides a modular, configuration-driven workflow that bridges the theoretical framework established in previous chapters with practical computational methods for investigating quantum coherence phenomena in complex molecular systems.

\section{Software Architecture and Design Philosophy}
\label{sec:software_architecture}

\noindent
The \texttt{qspectro2d} package follows a modular design that separates concerns across distinct functional domains. The architecture enables efficiency through parallel processing and flexible configuration management. The core design is built around a \textbf{configuration-driven workflow} where YAML-based parameter specification decouples physical models from computational implementation. This approach ensures that \textbf{modular components} including atomic systems, laser pulses, environmental baths, and spectroscopic calculations remain independently configurable. The framework provides \textbf{scalable execution} with support for both local testing and high-performance computing (HPC) batch processing, complemented by \textbf{comprehensive post-processing} through automated workflows for averaging, Fourier transforms, and visualization.

\noindent
The package structure comprises five main submodules: \texttt{core} (fundamental system components), \texttt{spectroscopy} (calculation engines), \texttt{config} (parameter management), \texttt{utils} (data handling utilities), and \texttt{visualization} (plotting tools).

\section{Configuration-Driven Simulation Workflow}
\label{sec:configuration_workflow}

\noindent
The simulation framework employs YAML configuration files to specify all physical and computational parameters, enabling reproducible plotes. A typical configuration defines the \textbf{atomic system} through the number of atoms, transition frequencies, geometric arrangement, and inhomogeneous broadening parameters. \textbf{Laser parameters} encompass pulse amplitudes, durations, carrier frequencies, and rotating wave approximation settings, while \textbf{environmental coupling} is characterized by bath type (Ohmic, Drude-Lorentz), temperature, coupling strength, and cutoff frequencies. Finally, \textbf{computational settings} specify the ODE solver choice (Lindblad vs. Bloch-Redfield), signal types, and time windows.

\noindent
The workflow consists of three main execution phases: simulation (\texttt{calc\_datas.py}), post-processing (\texttt{process\_datas.py}), and visualization (\texttt{plot\_datas.py}). This separation enables efficient parameter studies where computationally expensive simulations are performed once, followed by rapid exploration of different analysis and visualization options.

\section{Third-Order Polarization and Phase Cycling}
\label{sec:third_order_polarization}

\noindent
In the numerical implementation, the electric field for multiple pulses is modeled following the formalism established in Chapter~\ref{chapter_spectroscopy}. The total electric field is expressed as

\begin{equation}
	\vec{E}(\vec{r},t) = \sum_{m=1}^{3} \vec{E}_m(\vec{r},t) + \mathrm{c.c.},
\end{equation}

where c.c. denotes the complex conjugate, and

\begin{equation}
	\vec{E}_m(\vec{r},t) = \vec{\chi}_m E'_m(t - \tau_m) \exp(-i (\omega_m t + \varphi_m)),
\end{equation}

with frequency $\omega_m = 2\pi \nu_m$, wavevector $\vec{k}_m$, and electric field amplitude $\vec{\chi}_m$. The envelope $E'_m(t - \tau_m)$, centered at $\tau_m$, is assumed to be Gaussian with FWHM $\tau_p$

\begin{equation}
	E'_m(t - \tau_m) = \exp\left( -\frac{4 \ln 2 (t - \tau_m)^2}{\tau_p^2} \right).
\end{equation}

\noindent
The core spectroscopic calculation implements phase-cycled third-order polarization measurement following the established four-wave mixing protocols detailed in Chapter~\ref{chapter_spectroscopy}, Section~\ref{subsec:phase_cycling}.
\noindent
The implementation computes individual third orderp olarization components by subtracting the phase-cycled individual signals from the total one:
$P_{\phi_1,\phi_2}^{(3)}(t) = P_{\text{total}}(t) - \sum_i P_i(t)$, where $P_{\text{total}}$ includes all pulses and $P_i$ represents evolution with only pulse $i$ active. This approach enables isolation of the desired third-order response while automatically handling pulse overlap effects.

\section{Open Quantum System Evolution}
\label{sec:oqs_evolution}

\noindent
The quantum system evolution is handled by the \texttt{SimulationModuleOQS} class, which provides a unified interface for different solver backends implementing the open quantum system dynamics developed in Chapter~\ref{chapter_open_quantum_systems}. The implementation supports both Lindblad master equation evolution (via QuTiP's \texttt{mesolve}) and Bloch-Redfield dynamics (\texttt{brmesolve}) with automatic handling of pulse overlap and time-dependent Hamiltonians.

\noindent
For time-dependent pulse sequences, the total Hamiltonian takes the form established in the spectroscopy theory (Chapter~\ref{chapter_spectroscopy}):

\begin{equation}
	\label{eq:time_dependent_hamiltonian}
	H(t) = H_0 + \sum_{i} H_{\text{int},i}(t)
\end{equation}

\noindent
where $H_0$ is the bare system Hamiltonian and $H_{\text{int},i}(t)$ represents the interaction with pulse $i$, following the light-matter interaction formalism detailed in Chapter~\ref{chapter_spectroscopy}. The solver automatically detects pulse overlap regions and constructs appropriate piecewise evolution operators, ensuring accurate treatment of multi-pulse sequences regardless of timing complexity.

\section{Inhomogeneous Broadening and Statistical Averaging}
\label{sec:inhomogeneous_broadening}

\noindent
Real molecular systems exhibit inhomogeneous broadening due to environmental variations, as discussed in the spectroscopy fundamentals (Chapter~\ref{chapter_spectroscopy}). The implementation models this through Gaussian-distributed transition frequencies following the theoretical treatment:

\begin{equation}
	\label{eq:gaussian_broadening}
	\rho(\omega) = \frac{1}{\sigma\sqrt{2\pi}} \exp\left(-\frac{(\omega-\omega_0)^2}{2\sigma^2}\right)
\end{equation}

\noindent
where $\sigma = \Delta_{\text{FWHM}}/(2\sqrt{2\ln 2})$ relates the standard deviation to the experimentally relevant full width at half maximum, consistent with the broadening models established in Chapter~\ref{chapter_spectroscopy}. The sampling employs a rejection algorithm optimized for computational efficiency while maintaining statistical accuracy.

\section{Parallel Processing and Scalable Execution}
\label{sec:parallel_processing}

\noindent
The framework supports both local execution for testing and HPC cluster deployment for large parameter studies. The local workflow (\texttt{calc\_datas.py}) generates all combinations of coherence times and inhomogeneous samples, while the HPC workflow (\texttt{hpc\_batch\_dispatch.py}) distributes individual parameter combinations across compute nodes.

\noindent
For each combination of parameters, the averaged response is calculated following the inhomogeneous averaging procedure established in the spectroscopy theory (Chapter~\ref{chapter_spectroscopy}):

\begin{equation}
	\label{eq:averaged_response}
	\langle E_{k_s}(t_{\text{coh}}, t_{\text{det}}) \rangle = \frac{1}{N_{\text{inhom}}} \sum_{i=1}^{N_{\text{inhom}}} E_{k_s}^{(i)}(t_{\text{coh}}, t_{\text{det}})
\end{equation}

\noindent
where $N_{\text{inhom}}$ represents the number of inhomogeneous samples as defined in the broadening treatment (Chapter~\ref{chapter_spectroscopy}). The implementation utilizes Python's \texttt{ProcessPoolExecutor} for local parallel execution and integrates with SLURM-based job scheduling systems for HPC deployment, enabling efficient scaling from desktop testing to production-scale parameter sweeps.

\section{Post-Processing and Spectral Analysis}
\label{sec:post_processing}

\noindent
The post-processing pipeline (\texttt{process\_datas.py}) handles the conversion from time-domain signals to 2D spectra. \textbf{Data aggregation} groups individual parameter combination files by inhomogeneous sample and stacks them along coherence time dimensions. \textbf{Zero-padding} extends the time-domain data to improve frequency resolution according to:

\begin{equation}
	\label{eq:zero_padding}
	E_{\text{extended}}(t) = \begin{cases}
		E(t) & \text{for } t \in [0, T_{\text{orig}}] \\
		0 & \text{for } t \in (T_{\text{orig}}, T_{\text{extended}}]
	\end{cases}
\end{equation}

\noindent
Subsequently, \textbf{Fourier transformation} converts the data to frequency domain using Eq. \eqref{eq:rephasing_transform}.

\noindent
The frequency axes are expressed in wavenumber units ($10^4$ cm$^{-1}$) for direct comparison with experimental literature.

\section{Model System Implementation}
\label{sec:model_systems}

\noindent
The \texttt{AtomicSystem} class provides flexible construction of multi-atom quantum systems with configurable geometries, implementing the model systems described in Chapter~\ref{chapter_model_systems}. For the microtubule-inspired models central to this thesis, the implementation supports \textbf{cylindrical arrangements} where atoms are positioned on cylindrical surfaces with specified chain and ring numbers. \textbf{Excitation manifold truncation} restricts the system to single and double excitation subspaces with automatic basis construction, while \textbf{dynamic coupling} enables nearest-neighbor and long-range interactions based on geometric proximity. The framework also incorporates \textbf{inhomogeneous disorder} through site-specific frequency variations sampled from realistic distributions.

\noindent
The system construction automatically handles basis transformations between site and excitation manifold representations, enabling efficient evolution in the appropriate eigenbasis while maintaining physical interpretability of results.

%-------------------------------------------------------------------------------
%	BIBLIOGRAPHY
%-------------------------------------------------------------------------------

\printbibliography

\end{document}
% IDEA ABOUT HOW TO STRUCTURE THE THESIS
%https://chatgpt.com/share/686e2740-d700-8009-8b73-fc47a1730560